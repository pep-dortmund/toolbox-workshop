\subsection{Debug}

\begin{frame}[fragile]{\lstinline[texcsstyle=*\color{white}]+\\overfullrule+}
  \fbox{
    \begin{minipage}{10em}
      \mbox{fooooooooooooooooooooo}
    \end{minipage}
  }

  \vspace{\baselineskip}
  \begin{lstlisting}
    Overfull \hbox (14.97614pt too wide) in paragraph at lines 10--10
  \end{lstlisting}
  Wo genau ist die problematische Stelle?

  \vspace{\baselineskip}
  \begin{lstlisting}
    \setlength\overfullrule{5pt}
  \end{lstlisting}

  \setlength\overfullrule{5pt}
  \fbox{
    \begin{minipage}{10em}
      \mbox{fooooooooooooooooooooo}
    \end{minipage}
  }

  \vspace{\baselineskip}
  Zeilen, die über den Rand ragen, werden markiert.
\end{frame}

\begin{frame}[fragile]{\texttt{draft}}
  Entweder als Option für die Dokumentklasse
  \begin{lstlisting}
    \documentclass[…, draft, …]{…}
  \end{lstlisting}
  oder auch nur für ein Bild
  \begin{lstlisting}
    \includegraphics[draft, height=2cm]{logos/pep.pdf}
  \end{lstlisting}
  \includegraphics[draft, height=2cm]{logos/pep.pdf}

  Vorteile:
  \begin{itemize}
    \item Ränder des Bilds sind sichtbar
    \item Bild muss nicht existieren (Größe stimmt dann aber nicht)
  \end{itemize}
\end{frame}

\AddToShipoutPictureFG*{\ShowFramePicture}
\begin{frame}[fragile]{\texttt{showframe}}
  Manchmal möchte man den Textbereich auf der Seite grafisch sehen.

  Das geht mit
  \begin{lstlisting}
    \usepackage{showframe}
  \end{lstlisting}
\end{frame}

\directlua{lvd = require("lua-visual-debug")}
\AtBeginShipoutNext{\directlua{lvd.show_page_elements(tex.box["AtBeginShipoutBox"])}}
\begin{frame}[fragile]{\texttt{lua-visual-debug}}
  Manchmal möchte man sicher gehen, dass Sachen ausgerichtet sind oder die richtige Größe haben.
  
  Dabei hilft
  \begin{lstlisting}
    \usepackage{lua-visual-debug}
  \end{lstlisting}

  \vspace{2\baselineskip}
  Hier noch eine Gleichung
  \begin{equation}
    \int_0^\infty \mathup{e}^{-x^2} \, \mathup{d}x = \frac{\sqrt{\mathup{\pi}}}{2}
  \end{equation}
\end{frame}
