\section{Tricks}

% \begin{frame}[fragile]{\texttt{\textbackslash OverfullCenter}}
%   \begin{lstverbatim}
%   \rlap{\includegraphics[height=0.3cm,width=\textwidth+25pt]{logos/tu.pdf}}
%   \end{lstverbatim}

%   \vspace{5pt}
%   \rlap{\includegraphics[height=0.3cm,width=\textwidth+25pt]{logos/tu.pdf}}

%   \vspace{5pt}
%   Bild oder Tabelle ist zu Breit, passt aber auf die Seite.\\
%   Wie kriegt man es in die Mitte?

%   \vspace{5pt}
%   \begin{lstverbatim}
%   \OverfullCenter{\includegraphics[height=0.3cm,width=\textwidth+25pt]{logos/tu.pdf}}
%   \end{lstverbatim}

%   \vspace{5pt}
%   \OverfullCenter{\includegraphics[height=0.3cm,width=\textwidth+25pt]{logos/tu.pdf}}

%   \begin{block}{Code}
%     \begin{lstverbatim}
%     \ExplSyntaxOn
%     \NewDocumentCommand \OverfullCenter {+m}
%     {
%       \noindent
%       \makebox[\linewidth]{#1}
%     }
%     \ExplSyntaxOff
%     \end{lstverbatim}
%   \end{block}
% \end{frame}

\begin{frame}[fragile]{\texttt{pdflscape}}
  Falls das Bild oder die Tabelle wirklich breiter als die Seite ist, ist vielleicht eine gedrehte Seite die Lösung.
  \begin{columns}[T]
    \column{0.5\textwidth}
    \begin{block}{benötigte Pakete}
      \begin{lstverbatim}
      \includepackage{pdflscape}
      \end{lstverbatim}
    \end{block}
    \begin{block}{\LaTeX-Code}
      \begin{lstverbatim}
      \begin{landscape}
        \begin{table}
          % . . .
        \end{table}
      \end{landscape}
      \end{lstverbatim}
    \end{block}
    \column{0.45\textwidth}
    \begin{itemize}
      \item Inhalt der \texttt{landscape}-Umgebung wird horizontal gesetzt (separate Seite)
      \item Seite wird im PDF-Reader horizontal angezeigt → schöner zu lesen
    \end{itemize}
  \end{columns}
\end{frame}

\begin{landscape}
  \begin{frame}
    \centering
    <insert wide table here>
  \end{frame}
\end{landscape}
