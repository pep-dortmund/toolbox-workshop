\section{Tricks}

% \begin{frame}[fragile]{\texttt{\textbackslash OverfullCenter}}
%   \begin{lstlisting}
%     \rlap{\includegraphics[height=0.3cm,width=\textwidth+25pt]{logos/tu.pdf}}
%   \end{lstlisting}
%
%   \vspace{5pt}
%   \rlap{\includegraphics[height=0.3cm,width=\textwidth+25pt]{logos/tu.pdf}}
%
%   \vspace{5pt}
%   Bild oder Tabelle ist zu Breit, passt aber auf die Seite.\\
%   Wie kriegt man es in die Mitte?
%
%   \vspace{5pt}
%   \begin{lstlisting}
%     \OverfullCenter{\includegraphics[height=0.3cm,width=\textwidth+25pt]{logos/tu.pdf}}
%   \end{lstlisting}
%
%   \vspace{5pt}
%   \OverfullCenter{\includegraphics[height=0.3cm,width=\textwidth+25pt]{logos/tu.pdf}}
%
%   \begin{block}{Code}
%     \begin{lstlisting}
%       \ExplSyntaxOn
%       \NewDocumentCommand \OverfullCenter {+m}
%       {
%         \noindent
%         \makebox[\linewidth]{#1}
%       }
%       \ExplSyntaxOff
%     \end{lstlisting}
%   \end{block}
% \end{frame}

\begin{frame}[fragile]{\texttt{pdflscape}}
  Falls das Bild oder die Tabelle wirklich breiter als die Seite ist, ist vielleicht eine gedrehte Seite die Lösung.
  \begin{columns}[onlytextwidth, t]
    \begin{column}{0.50\textwidth}
      \begin{Packages}
        \begin{lstlisting}
          \usepackage{pdflscape}
        \end{lstlisting}
      \end{Packages}
      \begin{block}{Code}
        \begin{lstlisting}
          \begin{landscape}
            \begin{table}
              % .
            \end{table}
          \end{landscape}
        \end{lstlisting}
      \end{block}
    \end{column}
    \begin{column}{0.46\textwidth}
      \begin{itemize}
        \item Inhalt der \texttt{landscape}-Umgebung wird horizontal gesetzt (separate Seite)
        \item Seite wird im PDF-Reader horizontal angezeigt → schöner zu lesen
      \end{itemize}
    \end{column}
  \end{columns}
\end{frame}

% \begin{landscape}
%   \begin{frame}
%     \centering
%     <insert wide table here>
%   \end{frame}
% \end{landscape}
