\subsection{Mathe: Expert}

\begin{frame}[fragile]{Alternative Mathe-Fonts}
  Manchmal braucht man einen Script-Font oder einen zweiten kalligraphischen.
  \vspace{1em}
  \begin{CodeExample}{0.83}
    \begin{lstlisting}
      \symcal{IA} \symbfcal{IA} % Latin Modern
      \symcal{IA} \symbfcal{IA} % XITS Math, StylisticSet=1
      \symscr{IA} \symbfscr{IA} % XITS Math
    \end{lstlisting}
  \CodeResult
    \Umathordordspacing\textstyle=4mu
                           $\symcal{I A} \symbfcal{I A}$ \\
    {\mathversion{xitsss1} $\symcal{I A} \symbfcal{I A}$} \\
    {\mathversion{xits}    $\symscr{I A} \symbfscr{I A}$}
  \end{CodeExample}

  \begin{block}{Mathe-Fonts einstellen}
    \begin{lstlisting}
      \setmathfont{XITS Math}[range={scr, bfscr}]
      \setmathfont{XITS Math}[range={cal, bfcal}, StylisticSet=1]
    \end{lstlisting}
  \end{block}
\end{frame}

\begin{frame}[fragile]{\texttt{\textbackslash Re}, \texttt{\textbackslash Im}}
  \lstinline+\Re+ und \lstinline+\Im+ sehen nicht aus, wie erwartet:
  \vspace*{-1.2em}
  \begin{CodeExample}{0.5}
    \begin{lstlisting}
      \Re z    \Im z
    \end{lstlisting}
  \CodeResult
    \strut
    $\symRe z \qquad \symIm z$
  \end{CodeExample}

  \begin{lstlisting}
    \AtBeginDocument{ % wird bei \begin{document} ausgeführt
      \let\symRe=\Re  % werden sonst wieder von unicode-math überschrieben
      \RenewDocumentCommand \Re {}
      {
        \operatorname{Re}
      }
      \let\symIm=\Im
      \RenewDocumentCommand \Im {}
      {
        \operatorname{Im}
      }
    }
  \end{lstlisting}

  \vspace{-0.5em}
  Besser:
  \vspace*{-1.2em}
  \begin{CodeExample}{0.5}
    \begin{lstlisting}
      \Re z    \Im z
    \end{lstlisting}
  \CodeResult
    \strut
    $\Re z \qquad \Im z$
  \end{CodeExample}
\end{frame}

\begin{frame}[fragile]{Richtiges Spacing für \texttt{\textbackslash left}, \texttt{\textbackslash right}}
  \begin{Packages}
    \begin{lstlisting}
      \usepackage{mleftright}
    \end{lstlisting}
  \end{Packages}

  \vspace*{-\baselineskip}
  \begin{CodeExample}{0.5}
    \begin{lstlisting}
      \sin  \left( x  \right) y
      \sin \mleft( x \mright) y
    \end{lstlisting}
  \CodeResult
    \strut
    $\sin  \left( x  \right) y$ \\
    $\sin \mleft( x \mright) y$
  \end{CodeExample}

  Man kann natürlich eigene kurze Makros für \lstinline+\mleft+ und \lstinline+\mright+ definieren. \\
  Beispiel:\lstinline+\l+ und \lstinline+\r+ (Textbedeutungen beachten!).

  \vspace*{0.5em}
  \begin{lstlisting}
    \let\ltext=\l
    \RenewDocumentCommand \l {}
    {
      \TextOrMath{ \ltext }{ \mleft }
    }
    \let\raccent=\r
    \RenewDocumentCommand \r {}
    {
      \TextOrMath{ \raccent }{ \mright }
    }
  \end{lstlisting}
\end{frame}

\begin{frame}[fragile]{\texttt{\textbackslash DeclarePairedDelimiter}}
  \begin{itemize}
    \item Mit dem \texttt{mathtools}-Befehl \lstinline+\DeclarePairedDelimiter+ können Befehle erzeugen werden, die Symbole um Ausdrücke setzen.
    \item Automatische \lstinline+*+-Variante, die mitwächst.
    \item Automatisch richtiges Spacing
  \end{itemize}
  \begin{CodeExample}{0.7}
    \begin{lstlisting}
      % in Präambel
      \DeclarePairedDelimiter{\abs}{\lvert}{\rvert}
      \DeclarePairedDelimiter{\norm}{\lVert}{\rVert}

      % in Mathe:
      \abs{x} \abs*{\frac{1}{x}}
      \norm{\symbf{y}}

      \sin\abs*{\frac{1}{2}}
      \sin\left|\frac{1}{2}\right|
    \end{lstlisting}
  \CodeResult
    \vspace{5\baselineskip}
    \strut
    $\abs{x} \quad \abs*{\frac{1}{x}}$ \\
    $\norm{\symbf{y}}$ \\[\baselineskip]
    $\sin\abs*{\frac{1}{2}}$ \\[5pt]
    $\sin\left|\frac{1}{2}\right|$
  \end{CodeExample}
\end{frame}

\begin{frame}[fragile]{\texttt{\textbackslash bra}, \texttt{\textbackslash ket}, \texttt{\textbackslash braket}}
  Schonmal für Physik IV und Quantenmechanik vormerken.

  \begin{block}{In der Präambel}
    \begin{lstlisting}
      \DeclarePairedDelimiter{\bra}{\langle}{\rvert}
      \DeclarePairedDelimiter{\ket}{\lvert}{\rangle}
      % <name> <#arguments> <left> <right> <body>
      \DeclarePairedDelimiterX{\braket}[2]{\langle}{\rangle}{
        #1 \delimsize| #2
      }
    \end{lstlisting}
  \end{block}
  \begin{itemize}
    \item \lstinline+\delimsize+ gibt Größe der äußeren Klammern in \texttt{<body>}
  \end{itemize}

  \begin{CodeExample}{0.7}
    \begin{lstlisting}
      \bra{\Psi}
      \ket{\Psi}
      \braket*{\Psi_1}{\Psi_2}
    \end{lstlisting}
  \CodeResult
    \strut
    $\bra{\Psi}$ \\
    $\ket{\Psi}$ \\
    $\braket*{\Psi_1}{\Psi_2}$ \\
  \end{CodeExample}
\end{frame}

\begin{frame}[fragile]{\texttt{\textbackslash delimitershortfall}}
  Klammern wachsen nicht immer:
  \begin{CodeExample}{0.8}
    \begin{lstlisting}
      \left(  \left(  \left(  \left(
        x
      \right) \right) \right) \right)

      % in Präambel
      \setlength{\delimitershortfall}{-1sp}

      \left(  \left(  \left(  \left(
        x
      \right) \right) \right) \right)
    \end{lstlisting}
  \CodeResult
    \begin{CenterStrip}{3}
      $\left( \left( \left( \left( x \right) \right) \right) \right)$
    \end{CenterStrip}
    \\[4\baselineskip]
    \setlength{\delimitershortfall}{-1sp}
    \begin{CenterStrip}{3}
      $\left( \left( \left( \left( x \right) \right) \right) \right)$
    \end{CenterStrip}
  \end{CodeExample}
\end{frame}
