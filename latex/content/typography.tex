\subsection{Ein bisschen Typografie}
\begin{frame}[fragile]{Absatzauszeichnung}
  \begin{itemize}
    \item Zur Erinnerung: Leerzeile im Code erzeugt neuen Absatz
    \item Zwei Möglichkeiten: Einzug der ersten Zeile oder vertikaler Abstand
    \item Standard ist Einzug
    \item halbzeiliger vertikaler Abstand mit:
      \begin{block}{Klassenoption}
        \begin{lstlisting}
          \documentclass[parskip=half, ...]{scrartcl}
        \end{lstlisting}
      \end{block}
  \end{itemize}
\end{frame}

\begin{frame}[fragile]{\texttt{microtype}}
  \begin{itemize}
    \item Ihr werdet den Effekt kaum sehen
    \item \textbf{Das ist Absicht}
    \item Kleine Korrekturen, die das Schriftbild verbessern
    \item z.\,B. \enquote{-} etwas in den Rand hinein für homogenen Grauanteil
      \begin{Packages}
        \begin{lstlisting}
          \usepackage{microtype}
        \end{lstlisting}
      \end{Packages}
  \end{itemize}
\end{frame}

\begin{frame}[fragile]{Schönere Brüche im Text}
  \begin{Packages}
    \begin{lstlisting}
      \usepackage{xfrac}
    \end{lstlisting}
  \end{Packages}
  \begin{itemize}
    \item Problem: \lstinline+\frac{1}{2}+ zu hoch
    \item unschöne Alternative: 1/2
    \item schön: \lstinline+\sfrac{1}{2}+
  \end{itemize}
  \begin{CodeExample}{0.5}
    \begin{lstlisting}
      \sfrac{1}{2}
      \sfrac{$\symup{\pi}$}{2}
    \end{lstlisting}
  \CodeResult
    \includegraphics[scale=0.8]{figures/xfrac.pdf}
  \end{CodeExample}
\end{frame}

\begin{frame}[fragile]{Geschützte Leerzeichen}
  Es gibt Leerzeichen, an denen nicht umgebrochen werden soll. \\
  \begin{itemize}
    \item Zwischen Titel und Name
    \item Bei Referenzen
    \item Bei Datumsangaben
    \item Zweiteilige Ortsnamen
    \item Zweiteilige Abkürzungen (kleines Leerzeichen)
    \item Zwischen Zahl und Einheit (→ \texttt{siunitx})
  \end{itemize}
  Dafür gibt es die Tilde \lstinline+~+ (normaler Abstand) und \lstinline+\,+ (kleiner Abstand).
  \begin{CodeExample}{0.5}
    \begin{lstlisting}
      Prof.~Dr.~Dr.~Rhode
      Abbildung~\ref{fig:peplogo}
      2.~Oktober~2014
      St.~Helena
      z.\,B.
      \SI{3}{\newton\s}
    \end{lstlisting}
    \CodeResult
      \strut
      Prof.~Dr.~Dr.~Rhode \\
      Abbildung~\ref{fig:peplogo} \\
      2.~Oktober~2014 \\
      St.~Helena \\
      z.\,B. \\
      \SI{3}{\newton\s}
  \end{CodeExample}
\end{frame}

\begin{frame}[fragile]{Striche}
  Es gibt vier verschiedene Striche:
  \begin{CodeExample}{0.49}
    \begin{lstlisting}
      - $-$ -- ---
    \end{lstlisting}
  \CodeResult
    \strut
    - $-$ -- ---
  \end{CodeExample}

  \begin{description}[-- Halbgeviertstrich (en-dash)]
    \item[- Bindestrich]
      \begin{itemize}
        \item Bindestrich
        \item zwischen Doppelnamen der selben Person
          \begin{lstlisting}
            Levi-Civita-Symbol
          \end{lstlisting}
      \end{itemize}
    \item[-- Halbgeviertstrich (en-dash)]
      \begin{itemize}
        \item Gedankenstrich (wird mit Leerzeichen abgetrennt)
          \begin{lstlisting}
              Text -- oh, Gedankenstriche -- Text
          \end{lstlisting}
          \smallskip
        \item zwischen Namen von versch. Personen
          \begin{lstlisting}
            Maxwell--Boltzmann-Verteilung
          \end{lstlisting}
          \smallskip
        \item ist auch der Bis-Strich:
          \lstinline+1--10+ → sprich \enquote{1 bis 10}
      \end{itemize}
    \item[--- Geviertstrich (em-dash)]
      \begin{itemize}
          \item nicht im Deutschen genutzt, Gedankenstrich im Englischen
            \begin{lstlisting}
              text---oh, em-dashes---text
            \end{lstlisting}
      \end{itemize}
  \end{description}
\end{frame}

\begin{frame}[fragile]{Trennung bei Strichen}
  \vspace*{-2em}
  \begin{Packages}
    \begin{lstlisting}
      \usepackage[shortcuts]{extdash} % nach hyperref, bookmark
    \end{lstlisting}
  \end{Packages}

  Falls ein Wort Striche enthält, trennt \LaTeX\ ausschließlich an diesen.\\
  So ermöglicht man mehr Trennung:
  \vspace{-0.5em}
  \begin{CodeExample}{0.87}[Trennbare Striche]
    \begin{lstlisting}
      \-/ \-- \---
      Normal-Verteilung
      Normal\-/Verteilung
    \end{lstlisting}
  \CodeResult
    \strut
    - -- --- \\
    Normal-Verteilung \\
    Normal\-/Verteilung
  \end{CodeExample}

  So verhindert man die Trennung an den Strichen:
  \begin{lstlisting}
    \=/ \== \===
    $x$\=/Achse
  \end{lstlisting}
\end{frame}

\begin{frame}[fragile]{Silbentrennung}
  \begin{itemize}
    \item Manchmal kann \LaTeX\ ein Wort nicht richtig trennen
    \item Manche Fachwörter sollten nicht nach deutschen Regeln getrennt werden
  \end{itemize}
  \begin{block}{Trennung für Wort vorgeben}
    \begin{lstlisting}
      % Präambel
      \hyphenation{Dia-mag-ne-tis-mus hy-phen-ate hy-phen-a-tion}
      % statt Di-a-mag-ne-tis-mus

      hy\-phen\-ate % im Text
    \end{lstlisting}
  \end{block}
\end{frame}
