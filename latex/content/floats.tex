\section{Floating-Umgebungen}
\begin{frame}
    \frametitle{Gleitumgebungen}
    Zum setzen Elementen, die nicht zum Fließtext gehören, werden Gleitumgebungen genutzt. Diese werden automatisch an eine passende Stelle gesetzt.
    \begin{itemize}
        \item für Abbildungen und Tabellen
        \item Die Freiheit die \LaTeX \ beim setzen hat, kann mit optionalen Argumenten gestuert werden.
        \item h  (here), !h (noch strenger), t (top), b (bottom)
    \end{itemize}
\end{frame}
\begin{frame}[fragile]
    \frametitle{Bilder einbinden}
    \begin{block}{benötigte Pakete}
        \begin{lstverbatim}
        \usepackage{graphicx}
        \usepackage[labelfont=bf]{caption}
        \end{lstverbatim}
    \end{block}
    \begin{columns}[T]
        \begin{column}{0.6\textwidth}
            \begin{block}{\LaTeX-Code}
                \begin{lstverbatim}
                \begin{figure}
                    \centering
                    \includegraphics[width=\textwidth]{logos/pep.pdf}
                    \caption{Das Pep-Logo}
                    \label{fig:peplogo}
                \end{figure}
                \end{lstverbatim}
            \end{block}
        \end{column}
        \begin{column}{0.35\textwidth}
            \begin{block}{Ergebnis}
                \begin{figure}
                    \centering
                    \includegraphics[width=\textwidth]{logos/pep.pdf}
                    \caption{Das PeP-Logo}
                    \label{fig:peplogo}
                \end{figure}
            \end{block}
        \end{column}
    \end{columns}
\end{frame}

