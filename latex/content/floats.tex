\section{Gleitumgebungen}

\begin{frame}[fragile]{Gleitumgebungen}
  Zum Setzen von Elementen, die nicht zum Fließtext gehören, werden Gleitumgebungen genutzt.
  Diese werden automatisch an eine passende Stelle gesetzt.

  \begin{Packages}
    \begin{lstlisting}
      % Floats innerhalb eines Sections halten
      \usepackage[section, below]{placeins}
      \usepackage[labelfont=bf]{caption} % Captions schöner machen
    \end{lstlisting}
  \end{Packages}

  \lstinline+\FloatBarrier+ kann benutzt werden, um alle vorigen Floats zu setzen.
\end{frame}

\begin{frame}[fragile]{Bilder einbinden}
  \begin{Packages}
    \begin{lstlisting}
      \usepackage{graphicx}
      \usepackage{grffile}
    \end{lstlisting}
  \end{Packages}
  \begin{CodeExample}{0.60}
    \begin{lstlisting}
      \begin{figure}
        \centering
        \includegraphics[width=\textwidth]{logos/pep.pdf}
        \caption{Das Pep-Logo}
        \label{fig:peplogo}
      \end{figure}
    \end{lstlisting}
  \CodeResult
    \begin{figure}
      \centering
      \includegraphics[width=\textwidth]{logos/pep.pdf}
      \caption{Das PeP-Logo}
      \label{fig:peplogo}
    \end{figure}
  \end{CodeExample}
\end{frame}
