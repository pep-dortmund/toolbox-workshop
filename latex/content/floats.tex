\section{Gleitumgebungen}
\begin{frame}[fragile]{
  Gleitumgebungen
  \hfill
  \doc{http://mirrors.ctan.org/macros/latex/contrib/caption/caption-deu.pdf}{caption}
}
  \begin{itemize}
    \item Zum setzen von Elementen, die nicht Fließtext sind
    \item Hauptsächlich Grafiken und Tabellen
    \item Position wird von \LaTeX{} automatisch bestimmt
    \item Nicht auf früherer Seite als umgebender Text
    \item Bekommen meist \lstinline+\caption+ und \lstinline+\label+
  \end{itemize}
  \begin{Packages}
    \begin{lstlisting}
      % Floats innerhalb einer Section halten
      \usepackage[section, below]{placeins}
      \usepackage[…]{caption} % Captions schöner machen
    \end{lstlisting}
  \end{Packages}

  \lstinline+\FloatBarrier+ kann benutzt werden, um alle vorigen Floats zu setzen.
\end{frame}

\begin{frame}[fragile]{
  Bilder einbinden
  \hfill
  \doc{http://mirrors.ctan.org/macros/latex/required/graphics/grfguide.pdf}{graphicx}
}
  \begin{Packages}
    \begin{lstlisting}
      \usepackage{graphicx}
      \usepackage{grffile}
    \end{lstlisting}
  \end{Packages}
  \begin{CodeExample}{0.65}
    \begin{lstlisting}
      \begin{figure}
        \centering
        \includegraphics[width=\textwidth]{logos/pep.pdf}
        \caption{Das Pep-Logo.}
        \label{fig:peplogo}
      \end{figure}
    \end{lstlisting}
  \CodeResult
    \begin{figure}
      \centering
      \includegraphics[width=\textwidth]{logos/pep.pdf}
      \caption{Das PeP-Logo.}
      \label{fig:peplogo}
    \end{figure}
  \end{CodeExample}
  \vspace{5pt}
  \begin{itemize}
    \item Auch möglich: \lstinline+height=...+, \lstinline+scale=...+
    \item \lstinline+\caption+ endet immer mit einem Punkt.
  \end{itemize}
\end{frame}

\begin{frame}[fragile]{
  Subfigures
  \hfill
  \doc{ftp://ftp.rrzn.uni-hannover.de/pub/mirror/tex-archive/macros/latex/contrib/caption/subcaption.pdf}{subcaption}
}
  \begin{Packages}
    \begin{lstlisting}
    \usepackage{subcaption}
    \end{lstlisting}
  \end{Packages}
  \begin{EmulateArticle}
    \begin{figure}
      \centering
      \begin{subfigure}{0.48\textwidth}
        \centering
        \includegraphics[height=0.75cm]{logos/pep.pdf}
        \caption{PeP-Logo.}
        \label{fig:pep2}
      \end{subfigure}
      \begin{subfigure}{0.48\textwidth}
        \centering
        \includegraphics[height=0.75cm]{logos/tu.pdf}
        \caption{Das TU-Logo.}
        \label{fig:TU}
      \end{subfigure}
      \caption{Zwei Logos, Abbildung \subref{fig:TU}: das TU-Logo.}\label{fig:logos}
    \end{figure}
  \end{EmulateArticle}
\end{frame}

\begin{frame}[fragile]{Subfigures: Code}
  \vspace{-3pt}
  \begin{block}{Code}
    \begin{lstlisting}
      \begin{figure}
        \centering
        \begin{subfigure}{0.48\textwidth}
          \centering
          \includegraphics[height=0.75cm]{logos/pep.pdf}
          \caption{PeP-Logo.}
          \label{fig:pep2}
        \end{subfigure}
        \begin{subfigure}{0.48\textwidth}
          \centering
          \includegraphics[height=0.75cm]{logos/tu.pdf}
          \caption{Das TU-Logo.}
          \label{fig:TU}
        \end{subfigure}
        \caption{Zwei Logos, Abbildung \subref{fig:TU}: Das TU-Logo.}
        \label{fig:logos}
      \end{figure}
    \end{lstlisting}
  \end{block}
\end{frame}

\begin{frame}[fragile]{Referenzen}
  \begin{block}{Code}
    \begin{lstlisting}
      \section{Messung mit Apparatur 2}
      \label{sec:apparatur2}
      % .
      \section{Auswertung}
      Wie in \ref{sec:apparatur2} beschrieben, ...
    \end{lstlisting}
  \end{block}
  \begin{itemize}
    \item Auch für Gleichungen, Grafiken, Tabellen
    \item Für Übersichtlichkeit sollten Labels den Typ der Referenz nennen:
      \begin{description}
        \item[Sections]    \texttt{sec:}
        \item[Gleichungen] \texttt{eqn:}
        \item[Abbildungen] \texttt{fig:}
        \item[Tabellen]    \texttt{tab:}
      \end{description}
    \item Bei Gleichungen: \lstinline+\eqref+ statt \lstinline+\ref+ → setzt Klammern: \eqref{eqn:maxwell1}
    \item \lstinline+\label+ immer nach dem, worauf verwiesen wird
  \end{itemize}
\end{frame}

\begin{frame}[fragile]{\lstinline+\\ref+ vs. \lstinline+\\subref+}
  \begin{CodeExample}{0.47}
    \begin{lstlisting}
      In Abbildung \ref{fig:logos} sehen Sie zwei Logos.
      In Abbildung \ref{fig:pep2} sehen Sie das PeP-Logo.
      In Abbildung \subref{fig:pep2} sehen Sie das PeP-Logo.
      In \autoref{fig:pep2} sehen Sie das PeP-Logo.
    \end{lstlisting}
    \CodeResult
      \strut
      In Abbildung~\ref{fig:logos} sehen Sie zwei Logos. \\[\baselineskip]
      In Abbildung~\ref{fig:pep2} sehen Sie das PeP-Logo. \\[\baselineskip]
      In Abbildung~\subref{fig:pep2} sehen Sie das PeP-Logo.\\[\baselineskip]
      In Abbildung~\ref{fig:peplogo} sehen Sie das PeP-Logo.
  \end{CodeExample}
  \vspace{2em}
  \lstinline+\subref+ nur in \lstinline+\caption{…}+ zu Subfigures sinvoll.
\end{frame}

\begin{frame}[fragile]{Positionen der Gleitumgebungen}
  \begin{itemize}
    \item \LaTeX{} hat 4 Regionen, in die es Float-Umgebungen platziert
      \begin{description}
        \item[\texttt{h}] here, zwischen Text
        \item[\texttt{t}] top, oben auf einer Seite
        \item[\texttt{b}] bottom, unten auf einer Seite
        \item[\texttt{p}] page, eigene Seite nur für Floats
      \end{description}
    \item Standardmäßig nur \texttt{t,b,p} genutzt
    \item \alert{Nicht} empfohlen: Änderung mit optionalem Argument an Umgebung
    \item Änderung des Standards mit dem Paket \texttt{float}
  \end{itemize}

  \begin{Packages}
    \begin{lstlisting}
      \usepackage{scrhack} % nach \documentclass

      \usepackage{float}
      \floatplacement{figure}{htbp}
      \floatplacement{table}{htbp}
    \end{lstlisting}
  \end{Packages}
\end{frame}
