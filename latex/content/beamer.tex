\section{Präsentationen mit \LaTeX: \texttt{beamer}}

\begin{frame}[fragile]{
  \texttt{beamer}
  \hfill
  \doc{http://mirrors.ctan.org/macros/latex/contrib/beamer/doc/beameruserguide.pdf}{beamer}
}
  \begin{itemize}
    \item Dokumentenklasse für Präsentationen
    \item \lstinline+frame+-Umgebung erzeugt eine Folie
    \item Bei Nutzung mit \lstinline+fontspec+ und \lstinline+unicode-math+ muss die Option \texttt{professionalfonts} gesetzt werden.
    \item Aussehen wird durch \enquote{themes} gesteuert.
    \item Viele themes werden mit \TeX-Live mitgeliefert.
    \item Sehen leider alle fast gleich aus.
  \end{itemize}
\end{frame}

\begin{frame}[fragile]{Minimal-Beispiel}
  \begin{center}
    \begin{lstlisting}
      \documentclass[
        professionalfonts,
      ]{beamer}

      \usepackage{fontspec}
      \usepackage[
        math-style=ISO,
        bold-style=ISO,
        nabla=upright,
        partial=upright,
        sans-style=italic,
      ]{unicode-math}
      \setmathfont{Latin Modern Math}

      \begin{document}
        \begin{frame}{title}
          Hallo Welt!
        \end{frame}
      \end{document}
    \end{lstlisting}
  \end{center}
\end{frame}

\begin{frame}[fragile]{Mehrere Spalten}
  \begin{itemize}
    \item \texttt{columns}-Umgebung für Bereich mit mehreren Spalten
    \item Option \texttt{onlytextwidth} damit nichts in den Rand ragt
    \item Mögliche option für vertikale Ausrichtung der Spalten:
      \begin{description}
        \item[t] top, funktioniert nicht bei Bildern
        \item[c] center
        \item[b] bottom
        \item[T] wie \texttt{t}, funktioniert aber auch bei Bildern
      \end{description}
    \item \texttt{column}-Umgebung erzeugt Spalte, Breite ist Pflichtargument
  \end{itemize}
  \begin{center}
    \begin{lstlisting}
      \begin{columns}[onlytextwidth]
        \begin{column}{0.45\textwidth}
          Hallo
        \end{column}
        \begin{column}{0.45\textwidth}
          Welt
        \end{column}
      \end{columns}
    \end{lstlisting}
  \end{center}
\end{frame}

\begin{frame}[fragile]{Blöcke}
  \begin{itemize}
    \item (Zu?) Oft genutztes Element in \texttt{beamer}-Präsentationen
    \item Standardblöcke können nicht viel → \texttt{tcolorbox}
  \end{itemize}
  \begin{CodeExample}{0.5}
    \begin{lstlisting}
      \begin{block}{Titel}
        Block Body
      \end{block}

      \begin{exampleblock}{Titel}
        Block Body
      \end{exampleblock}

      \begin{alertblock}{Titel}
        Block Body
      \end{alertblock}
    \end{lstlisting}
  \CodeResult
    \begin{block}{Titel}
      Block Body
    \end{block}
    \begin{exampleblock}{Titel}
      Block Body
    \end{exampleblock}
    \begin{alertblock}{Titel}
      Block Body
    \end{alertblock}
  \end{CodeExample}
\end{frame}

\begin{frame}[fragile]{Nervige Buttons abschalten}
  \begin{center}
    \begin{lstlisting}
      \documentclass[…]{beamer}
      %  …
      %  packages here
      %  …

      \setbeamertemplate{navigation symbols}{}

      \begin{document}
        \begin{frame}{title}
          Hallo Welt!
        \end{frame}
      \end{document}
    \end{lstlisting}
  \end{center}
\end{frame}

\begin{frame}[fragile]{\texttt{siunitx} mit beamer}
  \begin{center}
    \begin{lstlisting}
      \documentclass[professionalfonts]{beamer}
      %  …
      %  packages here
      %  …

      \usepackage{siunitx}

      \AtBeginDocument{
        \sisetup{
          math-rm=\mathrm,
          math-micro=µ, % AltGr+m = MICRO SIGN, Unicode: U+00B5
        }
      }

      \begin{document}
        \begin{frame}{title}
          \SI{5}{\micro\ohm}
        \end{frame}
      \end{document}
    \end{lstlisting}
  \end{center}
\end{frame}
