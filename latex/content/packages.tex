\begin{frame}[fragile]{Standardpakete}
  Die hier aufgezählten Pakete sollten immer geladen werden, da sie wesentliche Funktionen bieten und wichtige Einstellungen vornehmen.
  \begin{columns}[T]
    \begin{column}{0.47\textwidth}
      \begin{block}{Paket}
        \begin{lstverbatim}
        \usepackage{fixltx2e}
        \usepackage[aux]{rerunfilecheck}

        \usepackage[main=ngerman]{babel}

        % mehr Pakete hier

        \usepackage[unicode,pdfusetitle]{hyperref}
        \usepackage{bookmark}
        \end{lstverbatim}
      \end{block}
    \end{column}
    \begin{column}{0.47\textwidth}
      \begin{block}{Funktion}
        \LaTeXe\ korrigieren. \\
        Warnung, falls nochmal kompiliert werden muss. \\
        Deutsche Spracheinstellungen für das Dokument. \\[2\baselineskip]
        Für Hyperlinks im Dokument (z.B. Inhaltsverzeichnis $\rightarrow$ Kapitel). \\
        Erweiterte Bookmarks im PDF.
      \end{block}
    \end{column}
  \end{columns}

  \vspace{5pt}
  Die Reihenfolge ist manchmal wichtig!
\end{frame}

\begin{frame}[fragile]{
  KOMA-Script-Klassen
  \hfill\doc{http://mirrors.ctan.org/macros/latex/contrib/koma-script/doc/scrguide.pdf}{KOMA-Skript}
}
  Stellt die wichtigen Klassen \texttt{scrartcl}, \texttt{scrreprt} und \texttt{scrbook} zur Verfügung.
  Sehr gute Vorlagen schon mit den Standardeinstellungen, schnell global mit Klassenoptionen anpassbar.
  \begin{block}{fürs Praktikum empfohlenene Einstellungen}
    \begin{lstverbatim}
    \documentclass[
      parskip=half,          % Absätze durch halbe Leerzeile
      captions=tableheading, % Spacing für Tabellenüberschriften
      headsepline,           % Linie zwischen Kopfzeile und Text
      titlepage=firstiscover % Titleseite ist Deckblatt
    ]{scrartcl}
    \end{lstverbatim}
  \end{block}
\end{frame}
