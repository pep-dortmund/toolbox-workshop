\section{Ausblick}

\begin{frame}
  \centering
  \only<1>{
    \includegraphics[height=0.95\textheight]{figures/engines-few.pdf}
    \hspace{1.15em}
  }
  \only<2,3>{
    \includegraphics[height=0.95\textheight]{figures/engines-many.pdf}
  }
  \begin{tikzpicture}[remember picture,overlay]
    \tikzset{shift={(current page.center)}}
    \node at (-4.3,2.3) {
      \TeX-Engines
    };
    \only<3>{
      \node (here) at (-4.4,-2.0) {
        Sie sind hier
      };
      \draw [->, >=Stealth] (here.east) -- ++(1.55,-0.85);
    }
  \end{tikzpicture}
\end{frame}

\begin{frame}[fragile]{Warum eigentlich \LuaTeX}
  \linespread{1.5}
  \begin{description}[Lua-Programmierung]
    \item[Unicode-Input]
      \begin{itemize}
        \item Bequem, äöüßêéè funktioniert einfach
      \end{itemize}
    \item[OTF-Fonts]
      \begin{itemize}
        \item Alle Fonts benutzen, die man auf dem Rechner hat
      \end{itemize}
    \item[Unicode-Math]
      \begin{itemize}
        \item Mathe-Input über Unicode
        \item Stichwort: Compose-Key (XCompose, Linux)
        \item Code lesbarer, Tippen schneller
        \item Mehr Font-Möglichkeiten
      \end{itemize}
    \item[Lua-Programmierung]
      \begin{itemize}
        \item \TeX-Programmierung ist nicht besonders einfach
        \item Manche Pakete bieten weitergehende Funktionen nur über Lua
      \end{itemize}
  \end{description}
\end{frame}

\begin{frame}[t]
  \centering
  \only<1>{
    \includegraphics[height=6.1em]{figures/formats-few.pdf}
    \hspace{0.5em}
  }
  \only<2,3>{
    \includegraphics[height=0.95\textheight]{figures/formats-many.pdf}
  }
  \begin{tikzpicture}[remember picture,overlay]
    \tikzset{shift={(current page.center)}}
    \node at (-4.3,3.5) {
      \TeX-Formate
    };
    \only<3>{
      \node (here) at (-4.7,-0.85) {
        Sie sind hier
      };
      \draw [->, >=Stealth] (here.east) -- ++(3.08,0);
    }
  \end{tikzpicture}
\end{frame}

\begin{frame}{Warum \LaTeX3?}
  \begin{itemize}
    \item \LaTeX3 existiert (noch) nicht
    \item \texttt{expl3} ist \LaTeX3 unter \LaTeXe
    \item \texttt{xpackages} sind Pakete, die auf \texttt{expl3} aufbauen und neue Möglichkeiten bieten
    \item \texttt{xparse} macht das schreiben neuer (auch komplizierter) Befehle sehr einfach
    \item viele Pakete benutzen jetzt schon \texttt{expl3} und \texttt{xparse}
  \end{itemize}
\end{frame}

\begin{frame}[fragile]{Möglichkeiten mit \LaTeX}
  \begin{description}[MusiXTeX, Lilypond]
    \item[\href{http://mirrors.ctan.org/macros/latex/contrib/koma-script/doc/scrguide.pdf}{\texttt{scrlettr2}}] Briefe
    \item[\href{http://mirrors.ctan.org/macros/musixtex/doc/musixdoc.pdf}{MusiXTeX}, \href{http://www.lilypond.org/}{Lilypond}] Notensatz
    \item[\texttt{IEEEtrantools}] Mächtigere Matheumgebungen
    \item[Poster] \texttt{beamerposter}, \texttt{tcolorbox}
    \item[todonotes] TODOs im Text, Liste am Ende, Platzhalter für Grafiken
  \end{description}
\end{frame}

\begin{frame}[fragile]
  \centering
  {\Huge \LaTeX:}

  \vspace{10pt}
  \centering
  \begin{BVerbatim}[gobble=4]
    \DeclareRobustCommand{\LaTeX}{%
      L\kern-.36em%
      {\sbox\z@ T%
        \vbox to\ht\z@{\hbox{%
          \check@mathfonts
          \fontsize\sf@size\z@
          \math@fontsfalse\selectfont A}%
          \vss}%
      }%
      \kern-.15em%
      \TeX}
  \end{BVerbatim}

  \begin{center}
    \Huge
    … alles klar?
  \end{center}
\end{frame}
