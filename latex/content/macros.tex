\subsection{Makros}

\begin{frame}[fragile]{Eigene \LaTeX-Kommandos}
  Nach 20 Mal \lstinline+\symup{e}+ oder \lstinline+\symup{i}+ schreiben hat man keine Lust mehr.

  \vspace{2em}
  \begin{CodeExplanation}{0.45}
    \begin{lstlisting}
      % in Präambel
      \usepackage{expl3}
      \usepackage{xparse}

      \ExplSyntaxOn

      \NewDocumentCommand \I {} {
        \symup{i}
      }

      \ExplSyntaxOff
    \end{lstlisting}
  \Explanation
    \strut \\[0.5\baselineskip]
    experimental \LaTeX3 \\[2\baselineskip]
    bequeme Syntax für Definition von Befehlen \\[1\baselineskip]
    Befehl \lstinline+\I+ definieren, keine Argumente \\
    Ergebnis von \lstinline+\I+ \\[2\baselineskip]
    Syntax wieder ausschalten, wichtig!
  \end{CodeExplanation}
\end{frame}

\begin{frame}[fragile]{
  \texttt{xparse}
  \hfill
  \doc{http://mirrors.ctan.org/macros/latex/contrib/l3packages/xparse.pdf}{xparse}
}
  \lstinline+\ExplSyntaxOn+
  \begin{itemize}
    \item Leerzeichen werden völlig ignoriert
    \item \lstinline+~+ gibt ein Leerzeichen
  \end{itemize}

  \lstinline+\NewDocumentCommand \Befehl {Argumente} { Code }+
  \begin{itemize}
    \item \lstinline+\Befehl+ sollte nicht vorher existieren
    \item Argumente: ab \texttt{1} nummeriert
      \begin{description}
        \item[m] (mandatory) Pflichtargument (in \lstinline+{}+)
        \item[O\{foo\}] optional mit Standardwert \texttt{foo} (in \lstinline+[]+)
      \end{description}
    \item Weitere Argumenttypen in der Doku
    \item Argument im Code mit \lstinline+#1+ usw. verwenden
    \item \lstinline+##+ gibt ein echtes \lstinline+#+
  \end{itemize}
\end{frame}

\begin{frame}[fragile]{Beispiel: \lstinline[texcsstyle=*\color{white}]+\\dif+}
  \begin{lstlisting}
    \NewDocumentCommand \dif {m}
    {
      \mathinner{\symup{d} #1}
    }
  \end{lstlisting}

  \begin{CodeExample}{0.65}<valign=center>
    \begin{lstlisting}
      \begin{equation}
        \int^{} \dif{x} \dif{^2 \symbf{y}} x^2
            |\symbf{y}|
      \end{equation}
    \end{lstlisting}
  \CodeResult
    \begin{equation}
      \int^{} \dif{x} \dif{^2 \symbf{y}} x^2 |\symbf{y}|
    \end{equation}
  \end{CodeExample}
  \vspace*{1em}

  Das Prinzip gilt auch für $\symup{D} x$, $\symup{\delta} x$, $\symup{\Delta} x$. \\
  Dabei sind $\symup{D}$, $\symup{\delta}$, $\symup{\Delta}$ gerade, weil sie keine Variablen sind.

  \begin{CodeExample}{0.5}
    \begin{lstlisting}
      \dif{x} \Dif{x} \del{x} \Del{x}
    \end{lstlisting}
  \CodeResult
    \strut
    $\dif{x} \Dif{x} \del{x} \Del{x}$
  \end{CodeExample}
\end{frame}

\begin{frame}[fragile]{Beispiel: \lstinline[texcsstyle=*\color{white}]+\\v+}
  \begin{lstlisting}
    \let\vaccent=\v              % alten Befehl kopieren
    \RenewDocumentCommand \v {}  % Befehl überschreiben
    {
      \TextOrMath{
        \vaccent                 % Textmodus
      }{
        \symbf                   % Mathemodus
      }
    }
  \end{lstlisting}

  \begin{CodeExample}{0.65}
    \begin{lstlisting}
      \v{a}
      \begin{equation}
        \int^{} \dif{x} \dif{^2 \v{y}} x^2 |\v{y}|
      \end{equation}
    \end{lstlisting}
  \CodeResult
    \strut
    \v{a}
    \begin{equation}
      \int^{} \dif{x} \dif{^2 \v{y}} x^2 |\v{y}|
    \end{equation}
  \end{CodeExample}
\end{frame}

\begin{frame}[fragile]{Beispiel: Umgebung}
  \begin{lstlisting}
    \NewDocumentEnvironment {CenterStrip} {O{\textwidth} m}
    {                                              % Code für \begin
      \begin{minipage}[c][#2\baselineskip][c]{#1}
    }{                                             % Code für \end
      \end{minipage}
      \ignorespacesafterend  % Einrückung von Text nach Umgebung vermeiden
      % #1 und #2 können auch hier benutzt werden
    }
  \end{lstlisting}

  \begin{CodeExample}{0.5}
    \begin{lstlisting}
      \begin{CenterStrip}{3}
        vertikal zentriert!
      \end{CenterStrip}
      \\[2\baselineskip]
      \hfill
      \begin{CenterStrip}
          [0.6\textwidth]{4}
        vertikal zentriert!
      \end{CenterStrip}
    \end{lstlisting}
  \CodeResult
    \begin{CenterStrip}{3}
      \strut
      vertikal zentriert!
    \end{CenterStrip}
    \\[2\baselineskip]
    \hfill
    \begin{CenterStrip}[0.6\textwidth]{4}
      \strut
      vertikal zentriert!
    \end{CenterStrip}
  \end{CodeExample}
\end{frame}

\begin{frame}[fragile]{Alt: \lstinline[texcsstyle=*\color{white}]+\\newcommand+}
  Alte Befehle, die man häufig trifft:
  \begin{lstlisting}
    \newcommand*\Befehl[Anzahl Argumente]{Code}
    \newcommand*\Befehl[Anzahl Argumente][Default]{Code}
    \newenvironment*{Umgebung}[Anzahl Argumente]{\begin-Code}{\end-Code}
  \end{lstlisting}
  \begin{itemize}
    \item Nur ein optionales Argument möglich, muss erstes Argument sein
    \item \lstinline+\end+-Code kann Argumente nicht benutzen
  \end{itemize}
\end{frame}
