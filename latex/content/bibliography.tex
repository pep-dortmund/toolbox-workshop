\section{Literaturverzeichnis}

\begin{frame}[fragile]{Literaturverzeichnis}
  \begin{itemize}
    \item Wichtiger Teil vieler Dokumente, für wissenschaftliche Texte zwingend
    \item \BibLaTeX\ und \texttt{biber} bieten eine sehr angenehme Arbeitsweise
    \item Auch für sehr große Referenzdatenbanken geeignet
    \item Es gibt viele unterschiedliche Stile
    \item Standardstil fürs Praktikum geeignet
    \item Referenzen in \texttt{.bib}-Dateien
  \end{itemize}
  \begin{block}{Neue Klassenoption}
    \begin{lstlisting}
      \documentclass[…, bibliography=totoc, …]{scrartcl}
    \end{lstlisting}
  \end{block}
\end{frame}

\begin{frame}{Die \BibTeX-Familie}
  \centering
  \vspace{0.025\textheight}
  \includegraphics[height=0.95\textheight]{figures/bibtex-many.pdf}
  \begin{tikzpicture}[remember picture,overlay]
    \tikzset{shift={(current page.center)}}
    \only<2>{%
      \node (here) at (-4.3,-2.7) {%
        Sie sind hier
      };
      \draw [->, >=Stealth] (here.east) -- ++(3.3,0);
    }
  \end{tikzpicture}
\end{frame}

\begin{frame}{Warum \texttt{biber}?}
  \begin{itemize}
    \item Unterstützt Unicode-Input
    \item Wird weiterentwickelt, zusammen mit \BibLaTeX
    \item Sortiert richtig, nach regeln der jeweiligen Sprache
    \item Kann noch viele weitere Formate außer \texttt{.bib} lesen
    \item Unterstützt alle Funktionen von \BibLaTeX
  \end{itemize}
\end{frame}

\begin{frame}[fragile]{\texttt{.bib}-Dateien}
  \begin{itemize}
    \item Datenbank eurer Literatur
    \item Textdateien
    \item Bib\LaTeX{} definiert viele Eintragstypen und dazugehörige Felder
    \item Typen haben Pflichtfelder und weitere optionale Felder
    \item Viele Felder fordern ein bestimmtes Format, z.\,B.\ Autorenlisten, Daten
      \begin{description}[Seitenzahlen]
        \item[Namensliste] \lstinline+Nachname1, Vorname1 and Nachname2, Vorname2 and others+ 
        \item[Datum] \lstinline+YYYY-MM-DD+ (ISO8601)
        \item[Seitenzahlen] \lstinline+X--Y+
      \end{description}
  \end{itemize}

  \begin{block}{Allgemeine Syntax von BibTeX-Einträgen}
    \begin{lstlisting}
      @TYPE{entryname,
        field1 = {value2},
        field2 = {value2},
      }
    \end{lstlisting}
  \end{block}


  Viele Dienste stellen Zitationen im \texttt{.bib}-Format zur Verfügung, zum Beispiel:
  \begin{center}
    \url{https://scholar.google.com}
  \end{center}

  \emph{Vorsicht: müssen oft noch angepasst werden.}
\end{frame}


\begin{frame}[fragile]{Anleitungen: \lstinline+@manual+}
  \begin{block}{Pflichtfelder}
    \texttt{title},
    \texttt{year} or \texttt{date}
  \end{block}

  \begin{block}{Beispiel}
    \lstinputlisting[firstline=56, lastline=60]{examples.bib}
  \end{block}
  \fullcite{fp01}
\end{frame}

\begin{frame}[fragile]{Journal-Artikel: \lstinline+@article+}
  \begin{block}{Pflichtfelder}
    \texttt{author},
    \texttt{title},
    \texttt{journal},
    \texttt{year} oder \texttt{date}
  \end{block}

  \begin{block}{Beispiel}
    \lstinputlisting[firstline=1, lastline=13]{examples.bib}
  \end{block}

  \fullcite{photoeffekt}
\end{frame}

\begin{frame}[fragile]{Bücher: \lstinline+@book+}
  \begin{block}{Pflichtfelder}
    \texttt{author},
    \texttt{title},
    \texttt{year} oder \texttt{date}
  \end{block}

  \begin{block}{Beispiel}
    \lstinputlisting[firstline=15, lastline=22]{examples.bib}
  \end{block}

  \fullcite{gelb1}
\end{frame}

\begin{frame}[fragile]{Abschlussarbeiten: \lstinline+@thesis+, \lstinline+@phdthesis+, \lstinline+@mastersthesis+}
  \begin{block}{Pflichtfelder}
    \texttt{author},
    \texttt{title},
    \texttt{type} (nur für \lstinline+thesis+),
    \texttt{institution},
    \texttt{year} oder \texttt{date}
  \end{block}

  \begin{block}{Beispiel}
    \lstinputlisting[firstline=24, lastline=35]{examples.bib}
  \end{block}
  \fullcite{phd_mnoethe}
\end{frame}

\begin{frame}[fragile]{Konferenz-Proceedings: \lstinline+@inproceedings+}
  \begin{block}{Pflichtfelder}
    \texttt{author},
    \texttt{title},
    \texttt{booktitle},
    \texttt{year} oder \texttt{date}
  \end{block}

  \begin{block}{Beispiel}
    \lstinputlisting[firstline=37, lastline=46]{examples.bib}
  \end{block}
  \fullcite{hillas}
\end{frame}

\begin{frame}[fragile]{Online-Quellen: \lstinline+@online+}
  \begin{block}{Pflichtfelder}
    \texttt{title},
    \texttt{url} oder \texttt{doi} oder \texttt{eprint}
  \end{block}

  \begin{block}{Beispiel}
    \lstinputlisting[firstline=48, lastline=54]{examples.bib}
  \end{block}
  \fullcite{toolbox-dualboot}
\end{frame}

\begin{frame}[fragile]{Software: \lstinline+@software+}
  \begin{block}{Pflichtfelder}
    \texttt{title},
    \texttt{year} or \texttt{date}
  \end{block}

  \begin{block}{Beispiel}
    \lstinputlisting[firstline=62, lastline=67]{examples.bib}
  \end{block}
  \fullcite{python}
\end{frame}

\begin{frame}[fragile]{Zu Software gehörige Veröffentlichungen}
  Viele Programmbibliotheken haben auch wissenschaftliche Veröffentlichungen,
  die zusätzlich zitiert werden sollten:

  Zum Beispie: \url{https://www.scipy.org/citing.html}

  \begin{block}{Beispiel}
    \footnotesize
    \lstinputlisting[firstline=69, lastline=85]{examples.bib}
  \end{block}

  \fullcite{numpy}
\end{frame}

\begin{frame}[fragile]{Closed Access und (Pr)e-prints}
  Viele wissenschaftliche Artikel sind leider nicht öffentlich einsehbar.

  Für manche gibt es sogenannte E-Prints oder Pre-Prints, zum Beispiel auf dem arXiv.

  \begin{block}{Beispiel}
    \footnotesize
    \lstinputlisting[firstline=88, lastline=107]{examples.bib}
  \end{block}

  \fullcite{higgs}
\end{frame}

\begin{frame}[fragile]{%
  \BibLaTeX{}%
  \hfill%
  \doc{http://mirrors.ctan.org/macros/latex/contrib/biblatex/doc/biblatex.pdf}{biblatex}
}
  \begin{Packages}
    \begin{lstlisting}
      \usepackage{biblatex} % nach babel
      \addbibresource{lit.bib}
    \end{lstlisting}
  \end{Packages}
  \begin{CodeExample}{0.60}[Zitieren]
    \begin{lstlisting}
      \cite{numpy}
      \cite[20]{numpy}
      \cite[1--3]{numpy}
      \cite{gelb1, fp01}
    \end{lstlisting}
  \CodeResult
    \cite{numpy} \\
    \cite[20]{numpy} \\
    \cite[1--3]{numpy} \\
    \cite{gelb1, fp01}
  \end{CodeExample}
  \begin{block}{Verzeichnis ausgeben}
    \begin{lstlisting}
      \nocite{numpy}     % ins Verzeichnis, obwohl nicht explizit zitiert
      \nocite{*}         % alles aus .bib ins Verzeichnis
      \printbibliography
    \end{lstlisting}
  \end{block}
\end{frame}

\begin{frame}{Literaturverzeichnis}
  \centering
  \pause
  \Huge ???
\end{frame}

\begin{frame}[fragile]{
  \texttt{biber}
  \hfill
  \doc{http://mirrors.ctan.org/biblio/biber/documentation/biber.pdf}{biber}
}
  Die Idee ist:
  \begin{enumerate}
    \item \BibLaTeX\ erstellt eine Liste der \texttt{.bib}-Dateien und der benötigten Referenzen \\
      → \texttt{.bcf}-Datei
    \item \texttt{biber} liest Anweisungen, liest \texttt{.bib}, sucht und sortiert Referenzen \\
      → \texttt{.bbl}-Datei
    \item \BibLaTeX\ liest \texttt{.bbl}, gibt Verzeichnis aus
  \end{enumerate}

  \vspace{10pt}
  Also:
  \begin{block}{Aufrufe mit Literaturverzeichnis}
    \begin{lstlisting}
       lualatex file.tex
       biber file.bcf
       lualatex file.tex
    \end{lstlisting}
  \end{block}
\end{frame}

\begin{frame}{Literaturverzeichnis}
  \nocite{*}
  \printbibliography[heading=none]
\end{frame}

\begin{frame}[fragile]{Stile}
  \begin{itemize}
    \item Standardstil ist \enquote{numeric}
    \item Häufig genutzte Alternative: \enquote{alphabetic}
    \item Kombination aus Autorenname und Jahr: z.B. [Oli07]
    \item Viele weitere Stile → Doku
    \item Setzen mit \texttt{style=…} als Option für \texttt{biblatex}
  \end{itemize}
  \begin{block}{Code}
    \begin{lstlisting}
      \usepackage[style=alphabetic]{biblatex}
    \end{lstlisting}
  \end{block}
\end{frame}
