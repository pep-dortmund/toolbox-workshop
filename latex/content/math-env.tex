\subsection{Mathe-Umgebungen}

\begin{frame}{
  Mathe-Umgebungen
  \hfill
  \doc{http://mirrors.ctan.org/macros/latex/required/amslatex/math/amsldoc.pdf}{amsmath}
}
  \begin{itemize}
    \item \texttt{amsmath} stellt Mathe-Umgebungen für alles was man so braucht zur Verfügung
      \item Alle Gleichungen werden automatisch nummeriert
      \item \texttt{*} nach dem Umgebungsnamen sorgt für unnumerierte Gleichung
      \item Unnumerierte Gleichungen sollten selten sein
  \end{itemize}
\end{frame}

\begin{frame}[fragile]{Die \texttt{equation}-Umgebung}
  \begin{CodeExample}{0.60}
    \begin{lstlisting}
      Es gilt
      \begin{equation}
        \nabla \cdot \symbf{E}
          = \frac{\rho}{\varepsilon_0} .
          \label{eqn:maxwell1}
      \end{equation}
      Schon Gauß hatte das Durchflutungsgesetz
      \eqref{eqn:maxwell1} aufgestellt.
    \end{lstlisting}
  \CodeResult
  \begin{minipage}[c][8\baselineskip][c]{\textwidth}
      Es gilt
      \begin{equation}
        \nabla \cdot \symbf{E}
          = \frac{\rho}{\varepsilon_0} .
          \label{eqn:maxwell1}
      \end{equation}
      Schon Gauß hatte das Durchflutungsgesetz \eqref{eqn:maxwell1} aufgestellt.
    \end{minipage}
  \end{CodeExample}

  \begin{itemize}
    \item Satzzeichen gehören in die \texttt{equation}-Umgebung!
    \item Gleichung ist grammatikalisch ein Substantiv
    \item Gleichungen müssen immer Teil eines vollständigen Satzes sein
  \end{itemize}
\end{frame}

\begin{frame}[fragile]{Die \texttt{gather}-Umgebung}
  \begin{itemize}
    \item Für mehrere Gleichungen
    \item \lstinline+\\+ erzeugt neue Zeile
      \begin{itemize}
        \item Kein \lstinline+\\+ nach der letzten Zeile!
      \end{itemize}
    \item Jede Zeile bekommt eine Gleichungsnummer
  \end{itemize}
  \begin{CodeExample}{0.54}
    \begin{lstlisting}
      \begin{gather}
        (a + b)^2 = a^2 + 2ab + b^2 \\
        (a - b)^2 = a^2 - 2ab + b^2 \\
        (a+b) \cdot (a-b) = a^2 - b^2
      \end{gather}
    \end{lstlisting}
  \CodeResult
    \begin{minipage}[c][5\baselineskip][c]{\textwidth}
      \begin{gather}
        (a + b)^2 = a^2 + 2ab + b^2 \\
        (a - b)^2 = a^2 - 2ab + b^2 \\
        (a + b) \cdot (a - b) = a^2 - b^2
      \end{gather}
    \end{minipage}
  \end{CodeExample}

  \vspace{1em}
  \begin{itemize}
    \item Abhängig vom Fall ist die \texttt{gather}-Umgebung grammatikalisch ein Substantiv oder eine Aufzählung
  \end{itemize}
\end{frame}

\begin{frame}[fragile]{Die \texttt{align}-Umgebung}
  \begin{itemize}
    \item Für mehrere Gleichungen, die aneinander ausgerichtet werden
    \item \lstinline+&+ steuert Ausrichtung
    \item \lstinline+\\+ erzeugt neue Zeile
    \item Jede Zeile bekommt eine Gleichungsnummer
  \end{itemize}
  \begin{CodeExample}{0.63}
    \begin{lstlisting}
      \begin{align}
        a         &= 1 & b           &= 2 \\
        a \cdot b &= 5 & \frac{a}{b} &= 0.5
      \end{align}
    \end{lstlisting}
  \CodeResult
    \begin{minipage}[c][4\baselineskip][c]{\textwidth}
      \begin{align}
        a         &= 1 & b           &= 2 \\
        a \cdot b &= 2 & \frac{a}{b} &= 0.5
      \end{align}
    \end{minipage}
  \end{CodeExample}
\end{frame}

\begin{frame}[fragile]{Die \texttt{split}-Umgebung}
  \begin{itemize}
    \item Um überlange Gleichungen auf zwei Zeilen aufzuteilen.
    \item Kommt in den anderen Umgebungen zum Einsatz
    \item \lstinline+&+ steuert Ausrichtung
    \item \lstinline+\\+ erzeugt neue Zeile
    \item Gemeinsame Gleichungsnummer
  \end{itemize}
  \begin{CodeExample}{0.55}
    \begin{lstlisting}
      \begin{equation}
        \begin{split}
          (a+b)^3 = {} & a^3 + 3a^2b \\
                       & + 3ab^2 + b^3
        \end{split}
      \end{equation}
    \end{lstlisting}
  \CodeResult
    \begin{minipage}[c][6\baselineskip][c]{\textwidth}
      \begin{equation}
        \begin{split}
          (a+b)^3 = {} & a^3 + 3a^2b \\
                       & + 3ab^2 + b^3
        \end{split}
      \end{equation}
    \end{minipage}
  \end{CodeExample}
\end{frame}
