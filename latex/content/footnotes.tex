\section{Fußnoten}

\begin{frame}[fragile]{Fußnoten}
  \begin{CodeExample}{0.5}
    \begin{lstlisting}
      In diesem Versuch benutzen wir
      PMTs\footnote{Photo-Multiplier-Tubes}.
    \end{lstlisting}
    \CodeResult
    In diesem Versuch benutzen wir PMTs\footnote{Photo-Multiplier-Tubes}.
  \end{CodeExample}
  \begin{itemize}
    \item Anpassung von Fußnoten mit dem Paket \texttt{footmisc}
  \end{itemize}
\end{frame}
\begin{frame}[fragile]{Fußnoten in Floats}
  \begin{alertblock}{Vorsicht bei Float-Umgebungen!}
    \begin{lstlisting}
      \begin{figure}
        \includegraphics[height=0.5cm]{pep.pdf}
        \caption[Bla]{Bla\footnotemark}
      \end{figure}
      \footnotetext{Unsinnige Caption.}
    \end{lstlisting}
  \end{alertblock}
  \vspace{-1pt}
  \begin{itemize}
    \item \lstinline+\footnotemark+ an der Stelle wo die Fußnote sein soll
    \item Bei \lstinline+\caption+ muss der Inhalt der Caption nocheinmal
        ohne \lstinline+\footnotemark+ in \lstinline+[ ]+ gegeben werden
    \item \lstinline+\footnotetext{...}+ außerhalb der Umgebung für den Text der Fußnote
  \end{itemize}
\end{frame}
