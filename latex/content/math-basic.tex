\begin{frame}[fragile]{\$\dots\$-Umgebung}
  Für mathematische Symbole, Variablen und kleine Formeln im Fließtext.

  \begin{columns}[T]
    \begin{column}{0.6\textwidth}
      \begin{block}{\LaTeX-Code}
        \begin{lstverbatim}
        foo $x$ bar
        $x^i$
        $x^12$ bzw. $x^{12}$ % Vorsicht
        $x_\text{max}$
        $x_\frac{1}{2}$ bzw. $x_{\frac{1}{2}}$
        $U(t) = U_0 \cdot \cos(\omega t)$
        \end{lstverbatim}
      \end{block}
    \end{column}
    \begin{column}{0.35\textwidth}
      \begin{block}{Ergebnis}
        foo $x$ bar \\
        $x^i$ \\
        $x^12$ bzw.\ $x^{12}$ \\
        $x_\text{max}$ \\
        Fehler! bzw.\ $x_{\frac{1}{2}}$ \\
        $U(t) = U_0 \cdot \cos(\omega t)$
      \end{block}
    \end{column}
  \end{columns}

  \vspace{10pt}
  Aufpassen, dass die Formel nicht zu hoch ist. \\
  Das sieht nicht gut aus $\frac{1^{\frac{1}{2}}}{2}$. \\
  Das auch nicht $\smash{\frac{1^{\frac{1}{2}}}{2}}$. Mehr Text für den Effekt. \\
  Mehr Text für den Effekt.
\end{frame}

% \begin{frame}[fragile]{Griechisch und mehr}
%   \begin{columns}[onlytextwidth,T]
%     \begin{column}{0.7\textwidth}
%       \begin{block}{Code}
%         \begin{lstverbatim}
%         \alpha \beta \gamma \delta \epsilon \zeta \eta \theta \iota \kappa \lambda \mu \nu \xi \omicron \pi \rho \sigma \tau \upsilon \phi \chi \psi \omega
%         \varepsilon \vartheta \varkappa \varpi \varrho \varsigma \varphi
%         \Alpha \Beta \Gamma \Delta \Epsilon \Zeta \Eta \Theta \Iota \Kappa \Lambda \Mu \Nu \Xi \Omicron \Pi \Rho \Sigma \Tau \Upsilon \Phi \Chi \Psi \Omega
%         \hbar \imath \jmath \ell \wp
%         \aleph \beth \gimel
%         \partial \eth \nabla \square \increment \infty
%         \end{lstverbatim}
%       \end{block}
%     \end{column}
%     \begin{column}{0.26\textwidth}
%       \begin{block}{Ergebnis}
%         \Umathordordspacing\textstyle=4mu
%         $\alpha \beta \gamma \delta \epsilon \zeta \eta$ \\
%         $\quad \theta \iota \kappa \lambda \mu \nu \xi$ \\
%         $\quad \omicron \pi \rho \sigma \tau \upsilon \phi$ \\
%         $\quad \chi \psi \omega$ \\
%         $\varepsilon \vartheta \varkappa \varpi \varrho$ \\
%         $\quad \varsigma \varphi$ \\
%         $\Alpha \Beta \Gamma \Delta \Epsilon \Zeta \Eta$ \\
%         $\quad \Theta \Iota \Kappa \Lambda \Mu \Nu \Xi$ \\
%         $\quad \Omicron \Pi \Rho \Sigma \Tau \Upsilon \Phi$ \\
%         $\quad \Chi \Psi \Omega$ \\
%         $\hbar \imath \jmath \ell \wp$ \\
%         $\aleph \beth \gimel$ \\
%         $\partial \eth \nabla \square \increment \infty$
%       \end{block}
%     \end{column}
%   \end{columns}
% \end{frame}

\begin{frame}{Operatoren}
\end{frame}

\begin{frame}{Indizes}
\end{frame}

\begin{frame}{Akzente}
\end{frame}

\begin{frame}{Funktionen}
\end{frame}

\begin{frame}{Große Operatoren}
\end{frame}

\begin{frame}[fragile]{Fonts}
    \begin{block}{Pakete}
        \begin{lstverbatim}
        \usepackage{unicode-math} % nach Mathe-Paketen und fontspec
        \end{lstverbatim}
    \end{block}
    \begin{columns}[t]
        \begin{column}{0.47\textwidth}
            \begin{block}{Latin Modern}
                \begin{lstverbatim}
                \setmathfont{Latin Modern Math}
                \end{lstverbatim}
            \end{block}
        \end{column}
        \begin{column}{0.47\textwidth}
            \begin{block}{Tex Gyre}
                \begin{lstverbatim}
                \setmathfont{Tex Gyre Pagella Math}
                \end{lstverbatim}
            \end{block}
        \end{column}
    \end{columns}
\end{frame}

\begin{frame}{Spaces}
\end{frame}

\begin{frame}[fragile]{
  Symbol-Sammlung
  \hfill\doc{http://mirrors.ctan.org/info/symbols/comprehensive/symbols-a4.pdf}{symbols-a4}
}
  Praktischer Link: \\
  \url{http://detexify.kirelabs.org/classify.html} \\
  (Symbol malen und LaTeX-Code angezeigt bekommen)
  \begin{columns}[T]
    \begin{column}{0.7\textwidth}
      \begin{block}{\LaTeX-Code}
        \begin{lstverbatim}
        \leq \geq \gg \ll \approx \propto
        \cdot \times \bar{x} \vec{x} \vec{\imath}
        \pm \mp
        \int_0^1 \sum_{i=1}^N \prod
        \iint \iiint \oint
        \end{lstverbatim}
      \end{block}
    \end{column}
    \begin{column}{0.25\textwidth}
      \begin{block}{Ergebnis}
        \begin{IEEEeqnarray*}{c}
          \leq \geq \gg \ll \approx \propto \\
          \cdot \times \bar{x} \vec{x} \vec{\imath} \\
          \pm \mp \\
          \int_0^1 \sum_{i=1}^N \prod \\
          \iint \iiint \oint
        \end{IEEEeqnarray*}
      \end{block}
    \end{column}
  \end{columns}
\end{frame}

\begin{frame}[fragile]{Mehr Mathematik}
  \begin{columns}[T]
    \column{0.7\textwidth}
    \begin{block}{\LaTeX-Code}
      \begin{lstverbatim}
      \begin{align*}
        \alpha, \beta, \gamma, \delta \\ 
        \Alpha, \Beta, \Gamma, \Delta \\
        \frac12 \quad \frac{3x^2 + 2x}{4x - 7} \\
        \sin(x) \quad \cos(x) \\
        \lim_{x \to \infty} \exp(-x) = 0 \\
        \sqrt[n]{a^2 + b^2} = c \\
        \langle x \rangle \; \text{vs.} \; < x >  \\
        \left(\frac{x^2}{y^3}\right)
      \end{align*}
      \end{lstverbatim}
    \end{block}
    \column{0.25\textwidth}
    \begin{block}{Resultat}
      \begin{align*}
        \alpha, \beta, \gamma, \delta \\ 
        \Alpha, \Beta, \Gamma, \Delta \\
        \frac12 \quad \frac{3x^2 + 2x}{4x - 7} \\
        \sin(x) \quad \cos(x) \\
        \lim_{x \to \infty} \exp(-x) = 0 \\
        \sqrt[n]{a^2 + b^2} = c \\
        \langle x \rangle \; \text{vs.} \; < x >  \\
        \left(\frac{x^2}{y^3}\right)
      \end{align*}
    \end{block}
  \end{columns}
\end{frame}

\begin{frame}[fragile]{Konventionen: Variablen, Zahlen, Einheiten, Indizes}
  \begin{itemize}
    \item Variablen werden kursiv gesetzt.
      Dies geschieht im Mathematikmodus automatisch.
    \item Einheiten werden aufrecht gesetzt und haben ein kleines Leerzeichen (\verb+\,+) Abstand zu ihrer Zahl.
      Am besten benutzt man hierfür immer \texttt{siunitx}.
    \item Die eulersche Zahl $\mathrm{e}$, das imaginäre $\mathrm{i}$ und das infinitesimale $\mathrm{d}$ werden ebenfalls aufrecht gesetzt.
      Im Mathematikmodus erreicht man dies mit
      \begin{lstverbatim}
      \mathrm{e}, \mathrm{d}, \mathrm{i}.
      \end{lstverbatim}
    \item Bestehen Indizes aus Text, wie min oder max, so wird dies ebenfalls aufrecht gesetzt.
      \begin{lstverbatim}
      x_\text{min}
      \end{lstverbatim}
    \item ein $\mathrm{d}x$ sollte durch ein kleines Leerzeichen (\verb+\,+) vom Integranden abgetrennt werden.
  \end{itemize}
\end{frame}
