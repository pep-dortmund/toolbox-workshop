\section{\texttt{Python} in \TeX}

\begin{frame}[fragile]{Code-Highlighting\hfill
  \doc{http://mirrors.ctan.org/macros/latex/contrib/minted/minted.pdf}{minted}}
  \begin{itemize}
    \item Im Praktikum sollen auch die Auswertungsskripte abgegeben werden.
    \item \texttt{minted} kann Code in \LaTeX\ darstellen, wie z.\,B.\ in diesen Folien
    \item und auch Skripte einbinden: \mintinline{latex}+\inputminted{python3}{plot.py}+
    \item Einstellungen über \mintinline{latex}+\setminted{+\dots\mintinline{latex}+}+
  \end{itemize}
\end{frame}

\begin{frame}[fragile]{Minted-Einstellungen}
  Folgende Einstellungen haben wir in der Protokoll-Vorlage gesetzt:

  \begin{tblr}{colspec={r l}}
    \mintinline{latex}+autogobble+ & Entfernt Leerzeichen am Anfang von Zeilen, wenn nicht benötigt \\
    \mintinline{latex}+breaklines+ & Lange Zeilen werden umgebrochen um auf die Seite zu passen \\
    \mintinline{latex}+linenos+ & Zeilennummern \\
    \mintinline{latex}+stripnl=true+ & Entfernt leere Zeilen am Anfang und Ende \\
    \mintinline{latex}+style=tango+ & Vorgefertigte Stilvorlage zum Highlighting \\
    \mintinline{latex}+baselinestretch=1.2+ & Etwas erhöhter Abstand zwischen den Zeilen \\
    \mintinline{latex}+bgcolor=LightGray+ & Hintegrundfarbe des Code-Bereichs \\
    \mintinline{latex}+fontsize=\footnotesize+ & Textgröße des Codes \\
  \end{tblr}
\end{frame}
