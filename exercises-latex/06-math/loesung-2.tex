\documentclass{scrartcl}

\usepackage[aux]{rerunfilecheck}

\usepackage{fontspec}

\usepackage[ngerman]{babel}

\usepackage{amsmath}
\usepackage{amssymb}
\usepackage{mathtools}

\usepackage[
  math-style=ISO,
  bold-style=ISO,
  sans-style=italic,
  nabla=upright,
  partial=upright,
  mathrm=sym,
]{unicode-math}

\usepackage[unicode]{hyperref}
\usepackage{bookmark}

\begin{document}

\section{Biot--Savart}

Das Magnetfeld $\symbf{B}$ am Ort $\symbf{r}$ eines stromdurchflossenen Leiters
ergibt sich zu
\begin{equation*}
  \symbf{B}(\symbf{r}) = \frac{\mu_0}{4 \symup{\pi}}
    \int_V \symbf{j}(\symbf{r}') \times
      \frac{\symbf{r} - \symbf{r}'}{\lvert \symbf{r} - \symbf{r}' \rvert^3}
      \, \symup{d}V' \, .
\end{equation*}
Hierbei bezeichnet $\symbf{j}$ die Stromdichte am Ort $\symbf{r}'$ und $\mu_0$
die magnetische Feldkonstante.

\section{Fehlerfortpflanzung}

\begin{equation*}
  \sigma_f = \sqrt{
    \sum_{i = 1}^N
      \left( \frac{\partial f}{\partial x_i} \sigma_i \right)^{\!\! 2}
  }
\end{equation*}

\section{Die vier Maxwellgleichungen}

\begin{minipage}{.48\textwidth}
  \begin{align}
    \nabla \cdot \symbf{E} &= \frac{\rho}{\varepsilon_0} \\
    \addtocounter{equation}{+1}
    \nabla \times \symbf{E} &= - \partial_t \symbf{B}
  \end{align}
\end{minipage}
\hfill
\begin{minipage}{.48\textwidth}
  \begin{align}
    \addtocounter{equation}{-2}
    \nabla \cdot \symbf{B} &= 0 \\
    \addtocounter{equation}{+1}
    \nabla \times \symbf{B} &= \mu_0 \symbf{j} + \mu_0 \varepsilon_0 \partial_t \symbf{E}
  \end{align}
\end{minipage}

\section{Wellengleichung}

Im Vakuum gelten $\rho = 0$ und $\symbf{j} = 0$, womit sich die Maxwellgleichungen zu
\begin{align*}
    \nabla \cdot \symbf{E} = 0 \label{eqn:max1} \stepcounter{equation}\tag{\theequation} \\
    \nabla \cdot \symbf{B} &= 0 \\
    \nabla \times \symbf{E} &= - \partial_t \symbf{B}
      \label{eqn:max3} \stepcounter{equation}\tag{\theequation} \\
    \nabla \times \symbf{B} &= \mu_0 \varepsilon_0 \partial_t \symbf{E}
      \label{eqn:max4} \stepcounter{equation}\tag{\theequation}
    \intertext{reduzieren.
      Nach erneuter Anwendung der Rotation auf~\eqref{eqn:max3} ergibt sich}
    \nabla \times \left( \nabla \times \symbf{E} \right)
      &= \nabla \times \left( - \partial_t \symbf{B} \right) \, .
    \intertext{Nach dem Satz von Schwarz lassen sich die partiellen Ableitungen vertauschen,
      was zu}
    \nabla \times \left( \nabla \times \symbf{E} \right)
      &= - \partial_t \! \left( \nabla \times \symbf{B} \right)
    \intertext{führt. Wir setzen auf der rechten Seite~\eqref{eqn:max4} ein:}
    \nabla \times \left( \nabla \times \symbf{E} \right)
      &= - \mu_0 \varepsilon_0 \partial_t^2 \symbf{E} \, .
    \intertext{Aus der linken Seite wird mit}
    \nabla \times \left( \nabla \times \symbf{E} \right)
      &= \nabla \cdot \left( \nabla \cdot \symbf{E} \right) - \increment \symbf{E}
    \shortintertext{und Ausnutzen von~\eqref{eqn:max1}}
    - \increment\symbf{E} &= -\mu_0 \varepsilon_0 \partial_t^2 \symbf{E} .
    \intertext{Dies ist die Wellengleichung für das elektrische Feld,
      in der sich die Lichtgeschwindigkeit}
    \symup{c} &= \frac{1}{\sqrt{\mu_0 \varepsilon_0}}
    \intertext{identifizieren lässt.
      Damit können wir}
    \left(\increment - \frac{1}{\symup{c}^2} \frac{\partial^2}{\partial t^2} \right)
      \! \symbf{E} &= 0
\end{align*}
schreiben.

\section{Wellengleichung}

Ebene Welle:
\begin{equation}
  \nabla^2 A - \frac{1}{\symup{c}^2} \frac{\partial^2}{\partial t^2} A = 0
\end{equation}
Eine Lösung:
\begin{equation}
  A = A_0 \exp(\mathrm{i} (\symbf{k} \symbf{x} - \omega t))
\end{equation}
Gruppen- und Phasengeschwindigkeit:
\begin{align}
  v_\text{Gr} &= \frac{\partial \omega}{\partial k} &
  v_\text{Ph} &= \frac{\omega}{k}
\end{align}

\section{Multipolentwicklung}

\begin{align*}
  \Phi(\symbf{r}) &= \frac{1}{4 \symup{\pi} \varepsilon_0} \left(
    \frac{Q}{r} + \frac{\symbf{r} \cdot \symbf{p}}{r^3}
    + \frac{1}{2} \sum_{k, l} Q_{k l} \frac{r_k r_l}{r^5} + \dotsb
  \right) \, ,
  \shortintertext{wobei}
  Q_{k l} &= \sum_{i\,=\,1}^n q_i
  \left( 3 r_{i k} r_{i l} - r_i^2 \symup{\delta}_{k l} \right) \, .
\end{align*}

\section{Jacobi-Matrix}

\begin{equation}
  \symbf{J} =
  \begin{pmatrix}
    \frac{\partial f_1}{\partial x_1} & \cdots & \frac{\partial f_1}{\partial x_n} \\
    \vdots                            & \ddots & \vdots                            \\
    \frac{\partial f_m}{\partial x_1} & \cdots & \frac{\partial f_m}{\partial x_n}
  \end{pmatrix}
\end{equation}

\section{Harmonischer Oszillator}

\begin{equation}
  \ddot{x} + 2 \gamma \dot{x} + \omega_0^2 x = 0
\end{equation}
Reelle Lösung:
\begin{align}
  x(t) &= \symup{e}^{-\gamma t} (A \cos(\omega t) + B \sin(\omega t)) \, ,
  \shortintertext{mit}
  \omega &= \sqrt{\omega_0^2 - \gamma^2} \, .
\end{align}

\end{document}
