\documentclass{scrartcl}

\usepackage[aux]{rerunfilecheck}

\usepackage{fontspec}

\usepackage[ngerman]{babel}

\usepackage{amsmath}
\usepackage{amssymb}
\usepackage{mathtools}

\usepackage[
  math-style=ISO,
  bold-style=ISO,
  sans-style=italic,
  nabla=upright,
  partial=upright,
]{unicode-math}

\usepackage[unicode]{hyperref}
\usepackage{bookmark}

\begin{document}

\section{Biot--Savart}

Das Magnetfeld $\symbf{B}$ am Ort $\symbf{r}$ eines stromdurchflossenen Leiters ergibt sich zu
\begin{equation}
  \symbf{B}(\symbf{r}) = \frac{\symup{\mu}_0}{4 \symup{\pi}}
    \int_V \symbf{j}(\symbf{r}') \times \frac{\symbf{r} - \symbf{r}'}{\lvert \symbf{r} - \symbf{r}' \rvert^3} \, \symup{d}V' .
\end{equation}
Hierbei bezeichnet $\symbf{j}$ die Stromdichte am Ort $\symbf{r}'$ und $\symup{\mu}_0$ die magnetische Feldkonstante.

\section{Fehlerfortpflanunzung}

\begin{equation}
  \sigma_f = \sqrt{
    \sum_{i = 1}^N
      \left( \frac{\partial f}{\partial x_i} \sigma_i \right)^{\!\! 2}
  }
\end{equation}

\section{Die vier Maxwellgleichungen}

\begin{align}
  \nabla \cdot  \symbf{E} &= \frac{\rho}{\symup{\varepsilon}_0} &
  \nabla \cdot  \symbf{B} &= 0 \\
  \nabla \times \symbf{E} &= - \partial_t \symbf{B} &
  \nabla \times \symbf{B} &= \symup{\mu}_0 \symbf{j}
    + \symup{\mu}_0 \symup{\varepsilon}_0 \partial_t \symbf{E}
\end{align}

\section{Wellengleichung}

Im Vakuum gelten $\rho = 0$ und $\symbf{j} = 0$, womit sich die Maxwellgleichungen zu
\begin{align}
  \nabla \cdot  \symbf{E} &= 0 \label{eqn:max1} \\
  \nabla \cdot  \symbf{B} &= 0 \label{eqn:max2} \\
  \nabla \times \symbf{E} &= - \partial_t \symbf{B} \label{eqn:max3} \\
  \nabla \times \symbf{B} &= \symup{\mu}_0 \symup{\varepsilon}_0 \partial_t \symbf{E} \label{eqn:max4}
\end{align}
reduzieren.
Nach erneuter Anwendung der Rotation auf \eqref{eqn:max3} ergibt sich
\begin{equation}
  \nabla \times \left( \nabla \times \symbf{E} \right) = \nabla \times \left( - \partial_t \symbf{B} \right) .
\end{equation}
Nach dem Satz von Schwarz lassen sich die partiellen Ableitungen vertauschen, was zu
\begin{equation}
  \nabla \times \left( \nabla \times \symbf{E} \right) = - \partial_t \! \left( \nabla \times \symbf{B} \right) .
\end{equation}
führt.
Wir setzen auf der rechten Seite \eqref{eqn:max4} ein:
\begin{equation}
  \nabla \times \left( \nabla \times \symbf{E} \right) = - \symup{\mu}_0 \symup{\varepsilon}_0 \partial_t^2 \symbf{E} .
\end{equation}
Aus der linken Seite wird mit
\begin{equation}
  \nabla \times \left( \nabla \times \symbf{E} \right) = \nabla \cdot \left( \nabla \cdot \symbf{E} \right) - \increment \symbf{E}
\end{equation}
und Ausnutzen von \eqref{eqn:max1}
\begin{equation}
  - \increment\symbf{E} = -\symup{\mu}_0 \symup{\varepsilon}_0 \partial_t^2 \symbf{E} .
\end{equation}
Dies ist die Wellengleichung für das elektrische Feld, in der sich die Lichtgeschwindigkeit
\begin{equation}
  \symup{c} = \frac{1}{\sqrt{\symup{\mu}_0 \symup{\varepsilon}_0}}
\end{equation}
identifizieren lässt.
Damit können wir
\begin{equation}
  \left(\increment - \frac{1}{\symup{c}^2} \frac{\partial^2}{\partial t^2} \right) \! \symbf{E} = 0
\end{equation}
schreiben.

\section{Wellengleichung}

Ebene Welle:
\begin{equation}
  \nabla^2 A - \frac{1}{c^2} \frac{\partial^2}{\partial t^2} A = 0
\end{equation}
Eine Lösung:
\begin{equation}
  A = A_0 \exp(\mathrm{i} (\symbf{k} \symbf{x} - \omega t))
\end{equation}
Gruppen- und Phasengeschwindigkeit:
\begin{align}
  v_\text{Gr} &= \frac{\partial \omega}{\partial k} &
  v_\text{Ph} &= \frac{\omega}{k}
\end{align}

\section{Multipolentwicklung}

\begin{equation}
  \Phi(\symbf{r}) = \frac{1}{4 \symup{\pi} \symup{\varepsilon}_0} \left(
    \frac{Q}{r} + \frac{\symbf{r} \cdot \symbf{p}}{r^3}
    + \frac{1}{2} \sum_{k, l} Q_{k l} \frac{r_k r_l}{r^5} + \dotsb
  \right) ,
\end{equation}
wobei
\begin{equation*}
  Q_{k l} = \sum_{i = 1}^n q_i \left( 3 r_{i k} r_{i l} - r_i^2 \symup{\delta}_{k l} \right)
\end{equation*}

\section{Jacobi-Matrix}

\begin{equation}
  \symbf{J} =
  \begin{pmatrix}
    \frac{\partial f_1}{\partial x_1} & \cdots  & \frac{\partial f_1}{\partial x_n} \\
    \vdots                            & \ddots  & \vdots                            \\
    \frac{\partial f_m}{\partial x_1} & \cdots  & \frac{\partial f_m}{\partial x_n}
  \end{pmatrix}
\end{equation}

\section{Harmonischer Oszillator}

\begin{equation}
  \ddot{x} + 2 \gamma \dot{x} + \omega_0^2 x = 0
\end{equation}
Reelle Lösung:
\begin{equation}
  x(t) = \symup{e}^{-\gamma t} (A \cos(\omega t) + B \sin(\omega t))
\end{equation}
mit
\begin{equation}
  \omega = \sqrt{\omega_0^2 - \gamma^2} .
\end{equation}

\end{document}
