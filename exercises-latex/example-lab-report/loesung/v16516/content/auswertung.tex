\section{Auswertung}
\label{sec:Auswertung}
Die für diesen Versuch relevanten physikalischen Größen für Kugel und Zylinder sind:
\begin{align}
  \label{eq:physical-properties}
  m_\text{Z} &= \input{mass_ball.tex} & m_\text{K} &= \input{mass_ball.tex}\nonumber\\
  r_{\text{Z},\text{i}} &= \input{../build/radius-inner_cylinder.tex} & r_\text{K} &= \input{radius_ball.tex}\\
  r_{\text{Z},\text{a}} &= \input{../build/radius-outer_cylinder.tex} &&\nonumber
\end{align}
Berechnet wurde aus diesen Größen das jeweilige, theoretische Trägheitsmoment
\begin{align}
  \label{eq:moments-inertia}
  I_\text{Z} &= \input{./../build/theoretical-I_cylinder.tex}\quad\text{ und } & I_\text{K} = \input{./../build/theoretical-I_ball.tex}.
\end{align}
Aus den aufgenommenen Videos der Versuche wurden die in \autoref{tab:all-measurements} folgenden Frameindizes extrahiert.
Die ebenfalls aufgeführte Zeitdauer $t$ wurde dann jeweils aus der aus dem Startframeindex $F_\text{i}$,
dem Endframeindex $F_\text{f}$ und der Kameraframerate $\symup{fps}=\input{framerate.tex}$ wie folgt 
berechnet
\begin{equation*}
  t = (F_\text{f} - F_\text{i})\cdot\symup{fps}^{-1}
\end{equation*}

\begin{table}
  \centering
  \caption{Alle aufgenommenen Werte, das heißt mit dreifach wiederholter Messungen je Höhe.}
  \label{tab:all-measurements}
  \input{build/table_all-measurements.tex}
\end{table}

Durch Mittelung der Messwerte für die selbe Höhe $h$ ergeben sich die Werte
in \autoref{tab:averaged-measurements}. Diese Werte werden für die folgenden
Auswertungsschritte verwendet.

\begin{table}
  \centering
  \caption{Für gleiche Starthöhe $h$ gemittelte Messwerte der Zeit $t$.}
  \label{tab:averaged-measurements}
  \input{build/table_averaged-measurements.tex}
\end{table}

\subsection{Bestimmung der Fallbeschleunigung $g$}

Für die bestimmung der Fallbeschleunigung wurden eine Ausgleichsfunktion der Form 
\eqref{eq:fit-function-g-ball} respektive \eqref{eq:fit-function-g-cylinder} an die  Messwerte aus \autoref{tab:averaged-measurements}
angepasst. Die Daten und Ausgleichsfunktion sind für die Kugel in \autoref{fig:fit-g-ball} 
und für den Zylinder in \autoref{fig:fit-g-cylinder} graphisch dargestellt.

\begin{figure}
  \centering
  \includegraphics{plot-g_ball.pdf}
  \caption{Dargestellt sind die für je eine Höhe gemittelten Messwerte aus \autoref{tab:averaged-measurements} für die Kugel,
  zusammen mit einer Ausgleichsfunktion der Form \eqref{eq:fit-function-g-ball} für die Bestimmung der Fallbeschleunigung $g$.
  Die Parameter der Ausgleichsrechnung sind in \eqref{eq:parameters-g_ball} angegeben.}
  \label{fig:fit-g-ball}
\end{figure}

\begin{figure}
  \centering
  \includegraphics{plot-g_cylinder.pdf}
  \caption{Dargestellt sind die für je eine Höhe gemittelten Messwerte aus \autoref{tab:averaged-measurements} für den Zylinder,
  zusammen mit einer Ausgleichsfunktion der Form \eqref{eq:fit-function-g-cylinder} für die Bestimmung der Fallbeschleunigung $g$.
  Die Parameter der Ausgleichsrechnung sind in \eqref{eq:parameters-g_cylinder} angegeben.}
  \label{fig:fit-g-cylinder}
\end{figure}

Die Parameter der Ausgleichsrechnung ergeben sich für die Kugel zu
\begin{align}
  \label{eq:parameters-g_ball}
  g_\text{K} &= \input{parameter-g_ball.tex}\quad\text{ und } & t_{0,\text{K}} &= \input{parameter-t0-g_ball.tex}
\intertext{und für den Zylinder}
  \label{eq:parameters-g_cylinder}
  g_\text{Z} &= \input{parameter-g_cylinder.tex}\quad\text{ und } & t_{0,\text{Z}} &= \input{parameter-t0-g_cylinder.tex}.
\end{align}

\subsection{Bestimmung der Trägheitsmomente $I$}

Für die bestimmung der Fallbeschleunigung wurden eine Ausgleichsfunktion der Form 
\eqref{eq:fit-function-I} an die Messwerte aus \autoref{tab:averaged-measurements}
angepasst. Die Daten und Ausgleichsfunktion sind für die Kugel in \autoref{fig:fit-I-ball} 
und für den Zylinder in \autoref{fig:fit-I-cylinder} graphisch dargestellt.


\begin{figure}
  \centering
  \includegraphics{plot-I_ball.pdf}
  \caption{Dargestellt sind die für je eine Höhe gemittelten Messwerte aus \autoref{tab:averaged-measurements} für den Kugel,
  zusammen mit einer Ausgleichsfunktion der Form \eqref{eq:fit-function-I} für die Bestimmung des Trägheitsmoments $I_\text{K}$.
  Die Parameter der Ausgleichsrechnung sind in \eqref{eq:parameters-I_ball} angegeben.}
  \label{fig:fit-I-ball}
\end{figure}


\begin{figure}
  \centering
  \includegraphics{plot-I_cylinder.pdf}
  \caption{Dargestellt sind die für je eine Höhe gemittelten Messwerte aus \autoref{tab:averaged-measurements} für den Zylinder,
  zusammen mit einer Ausgleichsfunktion der Form \eqref{eq:fit-function-I} für die Bestimmung des Trägheitsmoments $I_\text{Z}$.
  Die Parameter der Ausgleichsrechnung sind in \eqref{eq:parameters-I_cylinder} angegeben.}
  \label{fig:fit-I-cylinder}
\end{figure}

Die Parameter der Ausgleichsrechnung ergeben sich für die Kugel zu
\begin{align}
  \label{eq:parameters-I_ball}
  I_\text{K} &= \input{parameter-I_ball.tex}\quad\text{ und } &t_{0,\text{K}} &= \input{parameter-t0-I_ball.tex}
  \intertext{und für den Zylinder}
  \label{eq:parameters-I_cylinder}
  I_\text{Z} &= \input{parameter-I_cylinder.tex}\quad\text{ und } & t_{0,\text{Z}} &= \input{parameter-t0-I_cylinder.tex}.
\end{align}



