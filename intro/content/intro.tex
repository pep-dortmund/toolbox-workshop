\section{Einführung}

\begin{frame}{Ziele}
  \begin{center}
    \textcolor{vertexDarkRed}{\Huge Auf das Praktikum vorbereiten} \\[3ex]
    \Large Daten auswerten \hspace{2em} Plotten \hspace{2em} Fehlerrechnung
  \end{center}
\end{frame}

\begin{frame}{Ziele}
  \begin{center}
    \textcolor{vertexDarkRed}{\Huge Technische Fähigkeiten, die man in der Wissenschaft braucht} \\[3ex]
    \Large%
    Konkrete Probleme durch Programmieren lösen\\[2ex]
    Wiederholte Abläufe automatisieren\\[2ex]
    Versionskontrolle: Wieso? und Wie?\\[2ex]
    Kommandozeile
  \end{center}
\end{frame}

\begin{frame}{Ziele}
  \begin{center}
    \textcolor{vertexDarkRed}{\Huge Von Anfang an: Best Practices} \\[3ex]
    \Large%
    Spart Zeit und Nerven\hspace{2em}
    Verwenden von Dokumentation\\[2ex]
    Was sind die Standardwerkzeuge?
  \end{center}
\end{frame}

\begin{frame}
  \vspace*{1cm}
  \begin{center}
    \Huge Toolbox Workshop
  \end{center}
  \begin{tikzpicture}[remember picture,overlay]
    \tikzset{shift={(current page.center)}}
    \node at (4.3,1.3) {%
      \includegraphics[width=4cm]{logos/git.pdf}%
    };
    \node at (3.5,-2.3) {%
      \includegraphics[width=5cm]{logos/python.pdf}%
    };
    \node at (-3.2,-1.3) {%
      \includegraphics[width=6cm]{logos/matplotlib.png}%
    };
    \node at (0,2.5) {%
      \includegraphics[width=2cm]{logos/sympy.png}%
    };
    \node at (-4.3,1.3) {%
      \includegraphics[width=2cm]{logos/scipy.png}%
    };
    \node at (-3.5,-3) {%
      \includegraphics[width=4cm]{logos/numpy.png}%
    };
  \end{tikzpicture}
\end{frame}

\section{Umfrage}

\headlineframe{Ergebnisse der Umfrage}

\begin{frame}{Betriebssystem}
  \centering
  \includegraphics{figures/os.pdf}
\end{frame}

\begin{frame}{Programmierkenntnisse}
  \centering
  \includegraphics{figures/programming.pdf}
\end{frame}

\begin{frame}{Programmiersprachen}
  \centering
  \includegraphics{figures/languages.pdf}
%  \only<2->{%
%    \begin{tikzpicture}[overlay, remember picture, shift=(current page.center)]
%      \draw[vertexDarkRed, line width=2pt] (-5, -1.35) rectangle (-2.8, -0.75) node[below right] {\Large Keine Programmiersprache};
%    \end{tikzpicture}
%  }
\end{frame}

\begin{frame}{Interessen}
  \centering
  \includegraphics{figures/interest.pdf}
\end{frame}

\section{Ablauf}

\begin{frame}{Ablauf}
  \begin{description}
    \item[Montag] Programmieren mit Python
    \item[Dienstag] Erstellen von Plots / Auswerten
      \begin{itemize}
        \item NumPy
        \item matplotlib
      \end{itemize}
    \item[Mittwoch] Auswerten / Fehlerrechnung
      \begin{itemize}
        \item scipy
        \item uncertainties
      \end{itemize}
    \item[Donnerstag] Kommandozeile und Automatisierung
      \begin{itemize}
        \item Unix
        \item make
      \end{itemize}
    \item[Freitag] Versionskontrolle
      \begin{itemize}
        \item git
        \item Abschließende Übungen
      \end{itemize}
  \end{description}
\end{frame}

\section{Editoren}

\headlineframe{And now for something completely different…}

\headlineframe{Texteditoren}

\headlineframe{Was haben die mit diesem Kurs zu tun?}

\begin{frame}{Texteditoren}
  \begin{itemize}
    \item Viele Dateien, denen man in der Wissenschaft begegnet, enthalten (plain) text
      \begin{itemize}
        \item Paper/Arbeiten mit \LaTeX
        \item Programm-Code
        \item Config-Files
        \item Notizen
        \item Daten (csv, json, yaml, …)
        \item Emails
      \end{itemize}
    \item Es lohnt sich also, einen guten Texteditor zu wählen und den Umgang damit zu erlernen!
    \item Das spart auf lange Sicht Zeit und macht die Arbeit angenehmer
    \item Zwei Varianten: Terminal / GUI
  \end{itemize}
\end{frame}

\begin{frame}[c]{Textdateien und Unicode}
  Was ist eigentlich eine Textdatei?

  \begin{itemize}
    \item In einer Datei stehen immer Binärdaten in Bytes, 1 Byte = 8 Bit, 0-255
    \item Es gibt (gab) viele Varianten, Text in Binärdaten umzuwandeln (Encoding)
    \item Heute sollte immer Unicode enkodiert als \texttt{utf-8} verwendet werden
  \end{itemize}

  \begin{description}
    \item[Unicode] 
      \begin{itemize}
        \item Sammlung von Schriftzeichen, Buchstaben, Akzente, Emojis, ... 
        \item Aus allen Sprachen.
        \item Ordnet Zeichen \enquote{Codepoints} zu
        \item Beispiele: \texttt{LATIN SMALL LETTER A}: 97, \texttt{PILE OF POO}: 128169
      \end{itemize}
    \item[UTF-8] Encoding um Unicode-Text in Bytes zu speichern
  \end{description}

\end{frame}

\begin{frame}[c]{Zeilenende}

  Windows und Unix-Systeme verwenden unterschiedliche Konventionen für ein Zeilenende.

  \begin{description}
    \item[Unix] \mintinline{text}+\n+ / \texttt{LF} (Linefeed)
    \item[Windows] \mintinline{text}+\r\n+ / \texttt{CR LF} (Carriage Return + Linefeed).
  \end{description}

  VS Code erkennt auf allen Betriebssystemen welche Konvention im aktuellen File genutzt
  wird und behält sie bei.

  Empfehlung: immer Unix-Konvention nutzen

\end{frame}

\begin{frame}{Was muss ein Editor können?}
  In absteigender Wichtigkeit

  \begin{itemize}
    \item Zeilennummern
    \item Syntax-Highlighting
    \item Simple Autovervollständigung
    \item Plugins / Anpassbarkeit
    \item Linting (Warnhinweise für falschen Code)
    \item Komplexe Autovervollständigung (Snippets, Library-Funktionen)
  \end{itemize}
\end{frame}

\begin{frame}{Nano, Vim, GUIs}
  \begin{tabu}{@{} X[,C] @{} X[,C] @{} X[,C] @{}}
    \textbf{\large Nano} & \textbf{\Large Vim} & \textbf{\Large VS Code} \\
    \includegraphics[height=3.5cm]{figures/nano.png} &
    \includegraphics[height=3.5cm]{figures/vim.png} &
    \includegraphics[height=3.5cm]{figures/code.png} \\
    \begin{itemize}
      \item Einfacher Texteditor fürs Terminal
      \item Auf fast jedem Unix-System vorhanden
      \item Wenige Features, nicht erweiterbar
    \end{itemize}
    &
    \begin{itemize}
      \item Moden-basiert
      \item Erweiterbar
      \item Auf fast jedem Unix-System default
      \item Harter Einstieg
    \end{itemize}
    &
    \begin{itemize}
      \item GUI Editor von Microsoft
      \item Leichter zu bedienen
      \item Batteries included
      \item Viele nützliche Plugins
    \end{itemize}
  \end{tabu}
\end{frame}

\begin{frame}{Obligatory XKCD → \url{http://xkcd.com}}
  \centering
  \href{http://xkcd.com/378/}{\includegraphics[width=\textwidth]{figures/real_programmers.png}}
\end{frame}
