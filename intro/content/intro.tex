\section{Einführung}

\begin{frame}{Ziele}
  \begin{center}
    \textcolor{vertexDarkRed}{\Huge Auf das Praktikum vorbereiten} \\[3ex]
    \Large Daten auswerten \hspace{2em} Plotten \hspace{2em} Fehlerrechnung
  \end{center}
\end{frame}

\begin{frame}{Ziele}
  \begin{center}
    \textcolor{vertexDarkRed}{\Huge Technische Fähigkeiten, die man in der Wissenschaft braucht} \\[3ex]
    \Large%
    Konkrete Probleme durch Programmieren lösen\\[2ex]
    Wiederholte Abläufe automatisieren\\[2ex]
    Versionskontrolle: Wieso? und Wie?\\[2ex]
    Kommandozeile
  \end{center}
\end{frame}

\begin{frame}{Ziele}
  \begin{center}
    \textcolor{vertexDarkRed}{\Huge Von Anfang an: Best Practices} \\[3ex]
    \Large%
    Spart Zeit und Nerven\hspace{2em}
    Verwenden von Dokumentation\\[2ex]
    Was sind die Standardwerkzeuge?
  \end{center}
\end{frame}

\begin{frame}
  \vspace*{1cm}
  \begin{center}
    \Huge Toolbox Workshop
  \end{center}
  \begin{tikzpicture}[remember picture,overlay]
    \tikzset{shift={(current page.center)}}
    \node at (4.3,1.3) {%
      \includegraphics[width=4cm]{logos/git.pdf}%
    };
    \node at (3.5,-2.3) {%
      \includegraphics[width=5cm]{logos/python.pdf}%
    };
    \node at (-3.2,-1.3) {%
      \includegraphics[width=6cm]{figures/matplotlib.png}%
    };
    \node at (0,2.5) {%
      \includegraphics[width=2cm]{figures/sympy.png}%
    };
    \node at (-4.3,1.3) {%
      \includegraphics[width=2cm]{figures/scipy.png}%
    };
    \node at (-3.5,-3) {%
      \includegraphics[width=2cm]{figures/numpy.png}%
    };
  \end{tikzpicture}
\end{frame}

\section{Umfrage}

\headlineframe{Ergebnisse der Umfrage}

\begin{frame}{Betriebssystem}
  \centering
  \includegraphics{figures/os.pdf}
\end{frame}

\begin{frame}{Programmierkenntnisse}
  \centering
  \includegraphics{figures/programming.pdf}
\end{frame}

\begin{frame}{Interessen}
  \centering
  \includegraphics{figures/interest.pdf}
\end{frame}

\section{Ablauf}

\begin{frame}{Ablauf}
  \begin{description}
    \item[Montag] Programmieren mit Python / Erstellen von Plots
      \begin{itemize}
        \item Python
        \item NumPy
        \item matplotlib
      \end{itemize}
    \item[Dienstag] Erstellen von Plots / Auswerten / Fehlerrechnung
      \begin{itemize}
        \item matplotlib
        \item scipy
        \item uncertainties
      \end{itemize}
    \item[Mittwoch] Kommandozeile und Automatisierung
      \begin{itemize}
        \item Unix
        \item make
      \end{itemize}
    \item[Donnerstag] Versionskontrolle
      \begin{itemize}
        \item git
      \end{itemize}
    \item[Freitag] Ausführliche Übungen zu allen Themen
  \end{description}
\end{frame}

\section{Editoren}

\headlineframe{And now for something completely different…}

\headlineframe{Texteditoren}

\headlineframe{Was haben die mit diesem Kurs zu tun?}

\begin{frame}{Texteditoren}
  \begin{itemize}
    \item Viele Dateien, denen man in der Wissenschaft begegnet, enthalten (plain) text
      \begin{itemize}
        \item Paper/Arbeiten mit \LaTeX
        \item Programm-Code
        \item Notizen
        \item Daten (z.B. im CSV-Format)
        \item Emails
      \end{itemize}
    \item Es lohnt sich also, einen guten Texteditor zu wählen und den Umgang damit zu erlernen!
    \item Das spart auf lange Sicht Zeit und macht die Arbeit angenehmer
  \end{itemize}
\end{frame}

\begin{frame}{Vim, Emacs, Atom}
  \begin{tabu}{@{} X[,C] @{} X[,C] @{} X[,C] @{}}
    \includegraphics[width=0.75\linewidth]{figures/vim.png} &
    \includegraphics[width=0.75\linewidth]{figures/emacs.png} &
    \includegraphics[width=\linewidth]{figures/atom.jpg} \\
    \begin{itemize}
      \item Moden-basiert
      \item Erweiterbar
      \item Unix-Philosophie
      \item Auf jedem System vorhanden
    \end{itemize}
    &
    \begin{itemize}
      \item Unglaublich erweiterbar
      \item Enthält Mailprogramm
      \item Modifier-Tasten
      \item \enquote{Ein tolles Betriebssystem, dem nur ein guter Editor fehlt.}
    \end{itemize}
    &
    \begin{itemize}
      \item Neuer Editor von Github
      \item Leichter zu bedienen
      \item Viele nützliche Plugins
      \item Etwas langsam, da in Javascript geschrieben
    \end{itemize}
  \end{tabu}
\end{frame}

\begin{frame}
  \centering
  \includegraphics[height=0.97\textheight]{figures/editors.jpg}
\end{frame}

\begin{frame}{Obligatory XKCD → \url{http://xkcd.com}}
  \centering
  \href{http://xkcd.com/378/}{\includegraphics[width=\textwidth]{figures/real_programmers.png}}
\end{frame}
