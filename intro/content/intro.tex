\section{Einführung}


\begin{frame}{Motivation}
  \begin{center}
    \textcolor{vertexDarkRed}{\Huge Auf das Praktikum vorbereiten} \\[\baselineskip]
    \Large Daten:\hspace{2em} Abspeichern \hspace{2em} Auswerten \hspace{2em} Visualisieren \\ [\baselineskip]
    Zusammenarbeiten \hspace{2em} Protokoll verfassen\\
  \end{center}
\end{frame}

\begin{frame}{Motivation}
  \begin{center}
    \textcolor{vertexDarkRed}{\Huge Technische Fähigkeiten,\\ die man in der Wissenschaft braucht} \\[\baselineskip]
    \Large%
    Konkrete Probleme durch Programmieren lösen\\[\baselineskip]
    Wiederholte Abläufe automatisieren\\[\baselineskip]
    Versionskontrolle: Wieso? und Wie?\\[\baselineskip]
    Kooperation mit anderen an gemeinsamen Projekten\\[\baselineskip]
    Kommandozeile
  \end{center}
\end{frame}

\begin{frame}{Motivation}
  \begin{center}
    \textcolor{vertexDarkRed}{\Huge Von Anfang an: Best Practices} \\[\baselineskip]
    \Large%
    Spart Zeit und Nerven\\[\baselineskip]
    Verwenden von Dokumentation\\[\baselineskip]
    Erleichtert Zusammenarbeit mit anderen\\[\baselineskip]
    Was sind die Standardwerkzeuge?
  \end{center}
\end{frame}


\begin{frame}{Motivation}
  \begin{center}
    \textcolor{vertexDarkRed}{\Huge Offene, Reproduzierbare Wissenschaft} \\[\baselineskip]
    \Large%
    
    \begin{itemize}
      \item Große Teile der Wissenschaft stecken in einer \enquote{Replikationskrise}
      \item Gründe sind vielfältig, aber hauptsächlich
        \begin{itemize}
          \item Publikations-Bias (positive Ergebnisse werden veröffentlicht, Null-Resultate nicht)
          \item Zugrunde liegende Daten nicht verfügbar
          \item Genutzte Software nicht verfügbar
          \item Mangelhafte statistische Analyse
          \item Veröffentlichungen mit der vollen Information hinter \enquote{Paywalls}
          \item ...
        \end{itemize}
      \item Das muss besser gehen!
    \end{itemize}

  \end{center}
\end{frame}

\begin{frame}{Motivation}
  \begin{center}
    \textcolor{vertexDarkRed}{\Huge Open Science Grundprinzipien} \\[\baselineskip]

    \begin{tblr}{}
      1. & Öffentliche Daten & Open Data \\
      2. & Öffentlicher Quellcode & Open Source \\
      3. & Öffentliche Methodik & Open Methodology \\
      4. & Öffentlicher \enquote{Peer Review} Prozess & Open Peer Review \\
      5. & Öffentlicher Zugang zu Veröffentlichungen & Open Access \\
      6. & Öffentliche Bildungsangebote & Open Educational Resources \\
    \end{tblr}
  \end{center}
\end{frame}

\begin{frame}{Motivation}
  \begin{center}
    \textcolor{vertexDarkRed}{\Huge Reproduzierbarkeit} \\[\baselineskip]

    \begin{itemize}
      \item Öffentliche Software
      \item Öffentliche Daten
      \item Automatisierung so weit wie es irgendwie geht
      \item Versionskontrolle
    \end{itemize}
  \end{center}
\end{frame}

\begin{frame}{Motivation}
  \begin{center}
    \textcolor{vertexDarkRed}{\Huge FAIR Prinzipien für Daten} \\[\baselineskip]

    \begin{tblr}{}
      F & Findable \\
      A & Accessible  \\
      I & Interoperable \\
      R & Reusable \\
    \end{tblr}

    Kurz: Wissenschaftliche Daten müssen in wohldefinierten, offenen Datenformaten mit
    Metadaten, die den Inhalt des Datensatzes beschreiben, abgespeichert werden.

    Mehr infos hier: \url{https://www.go-fair.org/fair-principles/}
  \end{center}
\end{frame}

\begin{frame}{Toolbox-Workshop}
  \begin{center}
    \textcolor{vertexDarkRed}{\Huge Der Toolbox-Workshop}\\[\baselineskip]

    \begin{itemize}
      \item Einführung in einen zusammenpassenden Satz von Werkzeugen um reproduzierbare, offene Wissenschaft zu ermöglichen
      \item Das Praktikum soll im Kleinen die Grundlagen des wissenschaftlichen Arbeitens vermitteln \\
        $\Rightarrow$ Als Chance sehen, die hier vorgestellten Konzepte zu üben
      \item Spätestens essentiell bei Bachelor- und Masterarbeit
      \item Nützlich weit darüber hinaus
    \end{itemize}
  \end{center}
\end{frame}


\begin{frame}{Toolbox-Workshop}
  \begin{center}
    \textcolor{vertexDarkRed}{\Huge Der Toolbox-Workshop}\\[\baselineskip]
    \begin{itemize}
      \item Wir zeigen \emph{eine mögliche} Kombination von Tools 
      \item Für alle Bereiche gibt es andere Möglichkeiten mit Vor- und Nachteilen
      \item Die hier gezeigten Tools sind aber sehr weit verbreitet, auch außerhalb der Wissenschaft
    \end{itemize}
  \end{center}
\end{frame}


\begin{frame}
  \vspace*{1cm}
  \begin{center}
    \Huge Toolbox Workshop
  \end{center}
  \begin{tikzpicture}[remember picture,overlay]
    \tikzset{shift={(current page.center)}}
    \node at (4.3,1.3) {%
      \includegraphics[width=4cm]{logos/git.pdf}%
    };
    \node at (3.5,-2.3) {%
      \includegraphics[width=5cm]{logos/python.pdf}%
    };
    \node at (-3.2,-1.3) {%
      \includegraphics[width=6cm]{logos/matplotlib.png}%
    };
    \node at (0,2.5) {%
      \includegraphics[width=2cm]{logos/sympy.png}%
    };
    \node at (-4.3,1.3) {%
      \includegraphics[width=2cm]{logos/scipy.png}%
    };
    \node at (-3.5,-3) {%
      \includegraphics[width=4cm]{logos/numpy.png}%
    };
  \end{tikzpicture}
\end{frame}


\begin{frame}[fragile]{Python}
  \begin{center}
      \includegraphics[width=5cm]{logos/python.pdf}%
  \end{center}

  \begin{itemize}
    \item Höhere, stark und dynamisch typisierte, universelle, interpretierte Programmiersprache
    \item Beliebteste Programmiersprache in der Wissenschaft seit einigen Jahren
    \item Großes Ökosystem an wissenschaftlichen Bibliotheken
    \item Aber auch: Webentwicklung, Automatisierung, Systemprogrammierung, Spiele, etc.
    \item Einsteigerfreundlich, aber auch komplex
  \end{itemize}

  \begin{center}
    \mintinline{python}+print("Hello, World!")+
  \end{center}

  Weitere wichtige Programmiersprachen: C++, C, Java, Fortran, Rust, Julia, R
\end{frame}


\begin{frame}[fragile]{Python Installation}
  \begin{itemize}
    \item Es gibt (leider) viele Varianten Python und Bibliotheken für Python zu installieren
    \item Weit verbreitet in der Wissenschaft: die Paketmanager \enquote{conda} und \enquote{mamba}
    \item mamba ist eine wesentlich effizientere Implementierung des conda Tools, wird hoffentlich bald per default in conda integriert.\footnote{Siehe \url{https://www.anaconda.com/blog/conda-is-fast-now}}
  \end{itemize}
\end{frame}

\begin{frame}[fragile]{Isolierte Umgebungen}
  \begin{itemize}
    \item Wichtig: Isolierung von Software für verschiedene Projekte um Konflikte zu vermeiden und feste Versionen zu installieren (Reproduzierbarkeit!)
    \item conda / mamba ermöglichen es, isolierte, benannte Umgebungen zu erzeugen mit eigenen Versionen von Python und Bibliotheken
    \item Umgebungen müssen \enquote{aktiviert} werden um die Software darin zu nutzen:
      \begin{minted}{bash}
        $ which python
        /usr/bin/python
        $ mamba activate toolbox
        $ which python
        /home/maxnoe/.local/conda/envs/toolbox/bin/python
        $ mamba deactivate
      \end{minted}
    \item Umgebungen können in YAML Textdateien definiert werden
    \item Empfehlung: für jedes Projekt (zum Beispiel das Anfängerpraktikum) eine eigene Umgebung aufsetzen und \texttt{environment.yaml} Datei im Projekt speichern
  \end{itemize}
\end{frame}


\begin{frame}{Automatisierung}
  \begin{itemize}
    \item Reproduzierbarkeit erfordert Dokumentation und größtmögliche Automatisierung, da bei jedem Schritt Fehler gemacht werden können
    \item Wir zeigen die Software \enquote{GNU Make}
    \item Komplexe Abfolge von einzelnen Schritten wird in einer Textdatei beschrieben
    \item Führt alle notwendigen Schritte mit einem einzelnen Aufruf des \texttt{make} Befehls aus
    \item Wichtiger Zusammenhang mit der Kommandozeile:
      \begin{itemize}
        \item Makefiles verknüpfen Inputs mit Outputs über Kommandozeilen-Befehle
        \item Grafische Benutzeroberflächen sind zwar \enquote{schön} und \enquote{intuitiv} aber extrem schwierig zu automatisieren / reproduzierbar zu machen
      \end{itemize}
  \end{itemize}
\end{frame}


\begin{frame}{Versionskontrolle und Kooperation mit Git}
  \begin{itemize}
    \item Versionskontrolle mit Tools wie Git ist der zentrale Punkt von offener, reproduzierbarer Wissenschaft
    \item Genutzt in der Entwicklung quasi aller Open Source Software
    \item Ermöglicht Kooperation von mehreren Personen am gleichen Projekt
    \item Verküpft Änderungen mit Begründungen / Erklärungen
    \item Ermöglicht Rückkehr zu älteren Versionen
    \item Synchronisierung zwischen mehreren Rechnern
    \item Backup!
    \item Hosting-Provider wie GitHub und GitLab ermöglichen öffentliche und private Projekte
  \end{itemize}
\end{frame}


\section{Umfrage}

\headlineframe{Ergebnisse der Umfrage}

\begin{frame}{Betriebssystem}
  \centering
  \includegraphics{figures/os.pdf}
\end{frame}

\begin{frame}{Programmierkenntnisse}
  \centering
  \includegraphics{figures/programming.pdf}
\end{frame}

\begin{frame}{Programmiersprachen}
  \centering
  \includegraphics{figures/languages.pdf}
%  \only<2->{%
%    \begin{tikzpicture}[overlay, remember picture, shift=(current page.center)]
%      \draw[vertexDarkRed, line width=2pt] (-5, -1.35) rectangle (-2.8, -0.75) node[below right] {\Large Keine Programmiersprache};
%    \end{tikzpicture}
%  }
\end{frame}

\begin{frame}{Interessen}
  \centering
  \includegraphics{figures/interest.pdf}
\end{frame}

\section{Ablauf}

\begin{frame}{Ablauf}
  \begin{description}[Nächste Woche]
    \item[Montag] Programmieren mit Python
    \item[Dienstag] Erstellen von Plots / Auswerten
      \begin{itemize}
        \item NumPy
        \item matplotlib
      \end{itemize}
    \item[Mittwoch] Auswerten / Fehlerrechnung
      \begin{itemize}
        \item scipy
        \item uncertainties
      \end{itemize}
    \item[Donnerstag] Kommandozeile und Automatisierung
      \begin{itemize}
        \item Unix
        \item make
      \end{itemize}
    \item[Freitag] Versionskontrolle
      \begin{itemize}
        \item git
        \item Abschließende Übungen
      \end{itemize}
    \item[Nächste Woche] Verfassen wissenschaftlicher Texte mit \LaTeX{}
      \begin{itemize}
        \item Fließtext \& Mathematik
        \item Referenzen \& Literaturverzeichnis
        \item Integration mit dem Inhalt der ersten Woche
        \item Vollständige Übungen und Vorlage fürs Praktikum
      \end{itemize}
      
  \end{description}
\end{frame}

\section{Editoren}

\headlineframe{And now for something completely different…}

\headlineframe{Texteditoren}

\headlineframe{Was haben die mit diesem Kurs zu tun?}

\begin{frame}{Texteditoren}
  \begin{itemize}
    \item Viele Dateien, denen man in der Wissenschaft begegnet, enthalten (plain) text
      \begin{itemize}
        \item Paper/Arbeiten mit \LaTeX
        \item Programm-Code
        \item Config-Files
        \item Notizen
        \item Daten (csv, json, yaml, …)
        \item Emails
      \end{itemize}
    \item Es lohnt sich also, einen guten Texteditor zu wählen und den Umgang damit zu erlernen!
    \item Das spart auf lange Sicht Zeit und macht die Arbeit angenehmer
    \item Zwei Varianten: Terminal / GUI
  \end{itemize}
\end{frame}

\begin{frame}[c]{Textdateien und Unicode}
  Was ist eigentlich eine Textdatei?

  \begin{itemize}
    \item In einer Datei stehen immer Binärdaten in Bytes, 1 Byte = 8 Bit, 0-255
    \item Es gibt (gab) viele Varianten, Text in Binärdaten umzuwandeln (Encoding)
    \item Heute sollte immer Unicode enkodiert als \texttt{utf-8} verwendet werden
    \item Es gibt viele standardisierte Dateiformate, die auf Textdateien basieren\\
      json, yaml, toml, ecsv, ...
    \item Und weniger standardisierte aber trotzdem verbreitete Formate: \\
      csv, fixed width table, ...
  \end{itemize}

  \begin{description}
    \item[Unicode]
      \begin{itemize}
        \item Sammlung von Schriftzeichen, Buchstaben, Akzente, Emojis, ...
        \item Aus allen Sprachen.
        \item Ordnet Zeichen \enquote{Codepoints} zu
        \item Beispiele: \texttt{LATIN SMALL LETTER A}: 97, \texttt{PILE OF POO}: 128169
      \end{itemize}
    \item[UTF-8] Encoding um Unicode-Text in Bytes zu speichern
  \end{description}
\end{frame}

\begin{frame}[c]{Zeilenende}

  Windows und Unix-Systeme verwenden unterschiedliche Konventionen für ein Zeilenende.

  \begin{description}
    \item[Unix] \mintinline{text}+\n+ / \texttt{LF} (Linefeed)
    \item[Windows] \mintinline{text}+\r\n+ / \texttt{CR LF} (Carriage Return + Linefeed).
  \end{description}

  VS Code erkennt auf allen Betriebssystemen welche Konvention im aktuellen File genutzt
  wird und behält sie bei.

  Empfehlung: immer Unix-Konvention nutzen

\end{frame}

\begin{frame}{Was muss ein Editor können?}
  In absteigender Wichtigkeit

  \begin{itemize}
    \item Zeilennummern
    \item Syntax-Highlighting
    \item Simple Autovervollständigung
    \item Plugins / Anpassbarkeit
    \item Linting (Warnhinweise für falschen Code)
    \item Komplexe Autovervollständigung (Snippets, Library-Funktionen)
  \end{itemize}
\end{frame}

\begin{frame}{Nano, Vim, GUIs}
  \begin{tabu}{@{} X[,C] @{} X[,C] @{} X[,C] @{}}
    \textbf{\large Nano} & \textbf{\Large Vim} & \textbf{\Large Visual Studio Code} \\
    \includegraphics[height=3cm]{figures/nano.png} &
    \includegraphics[height=3cm]{figures/vim.png} &
    \includegraphics[height=3cm]{figures/code.png} \\
    \begin{itemize}
      \item Einfacher Texteditor fürs Terminal
      \item Auf fast jedem Unix-System vorhanden
      \item Wenige Features, nicht erweiterbar
    \end{itemize}
    &
    \begin{itemize}
      \item Moden-basiert
      \item Erweiterbar
      \item Auf fast jedem Unix-System default
      \item Harter Einstieg
    \end{itemize}
    &
    \begin{itemize}
      \item GUI Editor von Microsoft
      \item Leichter zu bedienen
      \item Batteries included
      \item Viele nützliche Plugins
    \end{itemize}
  \end{tabu}
\end{frame}

\begin{frame}{Obligatory XKCD → \url{http://xkcd.com}}
  \centering
  \href{http://xkcd.com/378/}{\includegraphics[width=\textwidth]{figures/real_programmers.png}}
\end{frame}
