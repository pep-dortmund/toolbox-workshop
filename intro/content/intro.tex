\section{Einführung}

\begin{frame}{Ziele}
  \begin{itemize}
    \item Auf das Praktikum vorbereiten
      \begin{itemize}
        \item Daten auswerten
        \item Plotten
        \item Fehlerrechnung
      \end{itemize}
    \item Fähigkeiten erlernen, die man als Wissenschaftler haben sollte
      \begin{itemize}
        \item Konkrete Probleme durch Programmierung lösen
        \item Wiederholte Abläufe automatisieren
        \item Versionskontrolle: Wieso? und Wie?
        \item Kommandozeile
      \end{itemize}
    \item Verwenden von Dokumentation
    \item Was sind die Standardwerkzeuge?
    \item Von Anfang an: Best Practices
  \end{itemize}
\end{frame}

\begin{frame}
  \begin{center}
    \Huge\color{TUgreen}Toolbox Workshop
  \end{center}
  \begin{tikzpicture}[remember picture,overlay]
    \tikzset{shift={(current page.center)}}
    \node at (4.3,1.3) {
      \includegraphics[width=4cm]{logos/git.pdf}
    };
    \node at (3.5,-2.3) {
      \includegraphics[width=5cm]{logos/python.pdf}
    };
    \node at (-3.2,-1.3) {
      \includegraphics[width=6cm]{figures/matplotlib.png}
    };
    \node at (0,2.5) {
      \includegraphics[width=2cm]{figures/sympy.png}
    };
    \node at (-4.3,1.3) {
      \includegraphics[width=2cm]{figures/scipy.png}
    };
    \node at (-3.5,-3) {
      \includegraphics[width=2cm]{figures/numpy.png}
    };
  \end{tikzpicture}
\end{frame}

\section{Umfrage}

\headlineframe{Umfrage}

\begin{frame}{Betriebssystem}
  \centering
  \includegraphics[height=0.9\textheight]{figures/os.pdf}
\end{frame}

\begin{frame}{Programmierkenntnisse}
  \centering
  \includegraphics[height=0.9\textheight]{figures/programming.pdf}
\end{frame}

\begin{frame}{Interessen}
  \centering
  \includegraphics[height=0.9\textheight]{figures/interest.pdf}
\end{frame}

\section{Ablauf}

\begin{frame}{Ablauf}
  \begin{description}
    \item[Montag] Programmieren und Auswerten mit Python
      \begin{itemize}
        \item Python
        \item NumPy, SciPy
      \end{itemize}
    \item[Dienstag] Erstellen von Plots und Fehlerrechnung
      \begin{itemize}
        \item matplotlib
        \item uncertainties
      \end{itemize}
    \item[Mittwoch] Kommandozeile und Automatisierung
      \begin{itemize}
        \item Unix
        \item make
      \end{itemize}
    \item[Donnerstag] Versionskontrolle
      \begin{itemize}
        \item git
      \end{itemize}
    \item[Freitag] Ausführliche Übungen
  \end{description}
\end{frame}

\section{Editoren}

\headlineframe{And now for something completely different…}

\headlineframe{Texteditoren}

\headlineframe{Was haben die mit diesem Kurs zu tun?}

\begin{frame}{Texteditoren}
  Ein guter Editor begleitet einen durch das Leben. \\[1em]
  Er wird Teil von einem, und wird ohne Gedanken gesteuert. \\[1em]
  Man verbringt den Großteil der Arbeitszeit im Editor. \\[1em]
  Man spart auf lange (und mittlere) Sicht unglaublich viel Zeit.
\end{frame}

\begin{frame}{Vim und Emacs}
  \begin{columns}
    \begin{column}{0.47\textwidth}
      \includegraphics[height=0.3\textheight]{figures/vim.png}
      \begin{itemize}
        \item Moden-basiert
        \item erweiterbar
        \item Unix-Philosophie
        \item auf jedem System vorhanden
      \end{itemize}
    \end{column}
    \begin{column}{0.47\textwidth}
      \includegraphics[height=0.3\textheight]{figures/emacs.png}
      \begin{itemize}
        \item unglaublich erweiterbar
        \item enthält Mailprogramm
        \item Modifier-Tasten
        \item \enquote{Ein tolles Betriebssystem, dem nur ein guter Editor fehlt.}
      \end{itemize}
    \end{column}
  \end{columns}
\end{frame}

\begin{frame}
  \centering
  \includegraphics[height=0.99\textheight]{figures/editors.jpg}
\end{frame}

\begin{frame}{Obligatory XKCD}
  \centering
  \href{http://xkcd.com/378/}{\includegraphics[width=\textwidth]{figures/real_programmers.png}}
\end{frame}
