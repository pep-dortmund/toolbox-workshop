\section{Betriebssysteme}

\headlineframe{Betriebssysteme}

\begin{frame}{Betriebssysteme}
    \begin{tblr}{
        colspec={X[c, m] X[c, m] X[c, m]},
        row{1}={font=\bfseries\Large}
    }
        \raisebox{-0.5\height}{\includegraphics[width=\linewidth]{build/figures/windows.png}} &
        \raisebox{-0.5\height}{\includegraphics[height=2cm]{build/figures/macos.png}} &
        \raisebox{-0.5\height}{\includegraphics[height=2cm]{build/figures/tux.jpg}} Linux \\
        Proprietäres Betriebssystem von Microsoft & Proprietäres Betriebssystem von Apple & Open Source Betriebssystem \emph{Kernel} \\
        Läuft auf den meisten Geräten & Läuft nur auf Apple Geräten & Läuft auf den meisten Geräten \\
        NT Kernel & Unix / BSD & Unix \\
        & & Viele verschiedene \emph{Distributionen}\\
    \end{tblr}

    Viele kommerzielle Software unterstützt nur Windows und/oder macOS.

    Man kann mehrere Betriebssysteme auf dem gleichen Rechner installieren. Nativ (\enquote{Dual-Boot}) oder in \enquote{virtuellen Maschinen}.

    Windows 10 \& 11 bringen das \emph{Windows Subsystem for Linux} mit, eine integrierte Linux VM.
\end{frame}

\begin{frame}{Linux-Distributionen}
    Eine Linux-Distribution kombiniert den Linux-Kernel mit weiterer Software. Hauptsächlich:
    \begin{itemize}
        \item Desktop-Umgebung(en)
        \item Paket-Manager und zugehörige Server mit Software
    \end{itemize}

    Es gibt viele \enquote{Familien} von Linux-Distributionen, die sich die gleichen oder ähnliche Tools teilen:
    \begin{center}
        \raisebox{-0.5\height}{\includegraphics[height=1.5cm]{build/figures/debian-openlogo.png}}
        \hspace{0.5cm}\raisebox{-0.5\height}{\includegraphics[height=1cm]{build/figures/ubuntu.png}} \\[1\baselineskip]

        \raisebox{-0.5\height}{\includegraphics[height=1.5cm]{build/figures/fedora.png}}~{\Huge Fedora}
        \hspace{0.5cm}\raisebox{-0.5\height}{\includegraphics[height=1.5cm]{build/figures/archlinux.png}}
    \end{center}

    Für den Einstieg empfehlen wir die aktuellen Versionen von Fedora oder Linux Mint.

    Übersicht über fast alle Distributionen: \url{https://distrowatch.com/}
\end{frame}

\begin{frame}{Linux-Distributionen}
    Warum wir (in der Mehrheit) Linux benutzen:
    \begin{itemize}
        \item Freiheit in der Auswahl der Software, nicht proprietär
        \item Flexibilität und Personalisierung im Aussehen und in der Handhabung
        \item Mehr Kontrolle über die eigenen Daten, \textbf{kein} Account oder
          Internetzugang bei der Installation notwendig
        \item \textbf{Kein} überwachendes oder \enquote{helfendes} KI System im OS
        \item Gewohnheit aus der (längeren) Zeit an der Uni (teilweise Interaktion mit Servern
            notwendig; diese sind meist ein Linux-artiges System)
    \end{itemize}
    Warum bei uns \textit{alles anders} aussieht:
    \begin{itemize}
        \item Selbstkonfigurierte, nicht unbedingt standardmäßige Desktop-Umgebung
        \item Unterschiedliche Distributionen (Arch, Fedora, PopOs, Endeavor, ...)
        \item \textbf{VIEL} investierte Zeit in das eigene System
    \end{itemize}

    Dennoch:

    Alle Funktionen, die wir euch zeigen, funktionieren auch bei euch.

\end{frame}
