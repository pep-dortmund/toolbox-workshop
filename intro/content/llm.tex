\section{LLM}

\headlineframe{Künstliche Intelligenz}

\begin{frame}{Bekannte Modelle}
    \begin{itemize}
        \item ChatGPT, Campus-AI, Gemini, Copilot, \dots
    \end{itemize}
    Das sind aber \textbf{L}arge \textbf{L}anguage \textbf{M}odelle.
    \begin{description}
        \item[KI] bezeichnet Systeme/Modelle, die Aufgaben ausführen können,
            die normalerweise menschliche Intelligenz erfordern
            (z.\,B.\ Mustererkennung, Entscheidungen treffen),
            indem sie statistische Zusammenhänge aus Daten lernen,
            anstatt auf expliziten physikalischen Gesetzen zu beruhen. 
        \item[LLM] sind eine spezielle Form von KI, die auf riesigen Textmengen trainiert wurden,
            um die Wahrscheinlichkeit des nächsten Wortes in einer Sequenz vorherzusagen.
            Dadurch kann es zusammenhängende Sprache erzeugen und Fragen beantworten.
    \end{description}
\end{frame}

\begin{frame}{Large Language Modelle}
    \textcolor{vertexDarkRed}{\textbf{Probleme:}}
    \begin{itemize}
        \item Sehr hoher Strom- und Wasserverbrauch.
        \item Die Ausgabe ist durchschnittlicher Code, durchschnittlicher Code ist schlecht.
        \item \textit{hidden Character attacks}, ungewünschte/schädliche Befehle über unsichtbare Zeichen
        \item Du musst den Code verstehen und (z.\,B.\ im Praktikum) erklären können.
        \item Was machst du, wenn die Antworten nicht funktionieren?
        \item Was machst du, wenn du keine LLMs nutzen darfst?
        \item Was machst du ohne Internet?
    \end{itemize}
\end{frame}

\begin{frame}{Large Language Modelle}
    \textcolor{vertexDarkRed}{\textbf{Sinnvoll nutzen:}}
    \begin{itemize}
        \item schlauere Suchmaschine \\
            \quad Du suchst nach einer Funktion, kommst aber nicht auf den Namen.
        \item Dokumentationen nachschlagen und mehr Beispiele anfragen
        \item Codeteile erklären lassen
        \item \textcolor{vertexDarkRed}{Jede Ausgabe hinterfragen!}
    \end{itemize}
\end{frame}