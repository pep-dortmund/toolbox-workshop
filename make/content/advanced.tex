\headlineframe{Is the cake a lie?}

\begin{frame}[fragile]{Vergleich: Kuchen backen}
  \begin{center}
    \inputminted{make}{example-files/Cake}
  \end{center}
\end{frame}

\headlineframe{Expert}

\begin{frame}[fragile]{Expert}
  Können mehrere unabhängige Auswertungen parallel ausgeführt werden? \\
  → Ja: \;\texttt{make -j4}\; (nutzt 4 Prozesse (\texttt{j}: jobs) gleichzeitig, Anzahl beliebig)

  Problem: Manchmal führt \texttt{make} Skripte gleichzeitig zweimal aus (hier \texttt{plot.py})
  \begin{center}
    \small
    \inputminted{make}{example-files/Advanced-0}
  \end{center}

  Lösung: Ein \texttt{make}-Feature (\texttt{v4.3}): \emph{grouped targets}
  \begin{center}
    \small
    \inputminted{make}{example-files/Advanced-1}
  \end{center}
   → Alle \texttt{targets} werden durch eine einzigen Durchlauf der \texttt{recipe} (gebündelt) erzeugt.
\end{frame}

\begin{frame}[fragile]{Expert}
  Wenn ein Skript sehr viele Dateien erzeugt kann die Liste der \texttt{targets} unübersichtlich lang werden.\\
  Diese \texttt{targets} werden außerdem in der Regel als \texttt{prerequisites} verwendet,
  z.~B. für die \texttt{recipe} der Berichtdatei  → Die Liste befindet sich sogar zweimal im \texttt{Makefile}

  \begin{center}
    \small
    \inputminted{make}{example-files/Advanced-2}
  \end{center}
  Lösung: Ein weiteres \texttt{make}-Feature: \emph{variables}
  \begin{center}
    \small
    \inputminted{make}{example-files/Advanced-3}
  \end{center}
\end{frame}

\begin{frame}[fragile]{Expert}
  Die Variablenliste kann auch weiter bearbeitet werden.
  Mit \texttt{addprefix build/} wird in diesem Beispiel der \texttt{build} Ordner
  an den Anfang jedes Dateipfades geschrieben.
  \texttt{addsuffix .pdf} hängt an jeden Dateinamen die Endung \texttt{.pdf} an.
  \begin{center}
    \small
    \inputminted{make}{example-files/Advanced-4}
  \end{center}
\end{frame}
