\headlineframe{Is the cake a lie?}

\begin{frame}[fragile]{Vergleich: Kuchen backen}
  \begin{center}
    \begin{minted}{make}
    Kuchen: Teig Backofen
        Ofen auf 140°C vorheizen
        Teig in Backform geben und in den Ofen schieben
        Kuchen nach 40 min herausnehmen

    Teig: Eier Mehl Zucker Milch Rumrosinen | Schüssel
        Eier schlagen
        Mehl, Zucker und Milch hinzugeben
        Rumrosinen unterrühren

    Rumrosinen: Rum Rosinen
        Rosinen in Rum einlegen
        Vier Wochen stehen lassen

    Schüssel:
        Rührschüssel auf den Tisch stellen, wenn nicht vorhanden

    clean:
        Kuchen essen
        Küche sauber machen und aufräumen
    \end{minted}
  \end{center}
\end{frame}

\begin{frame}[fragile]{Expert}
  Können mehrere unabhängige Auswertungen parallel ausgeführt werden? \\
  → Ja: \;\texttt{make -j4}\; (nutzt 4 Prozesse (\texttt{j}: jobs) gleichzeitig, Anzahl beliebig)

  Problem: Manchmal führt \texttt{make} Skripte gleichzeitig zweimal aus (hier \texttt{plot.py})
  \begin{center}
    \small
    \begin{minted}{make}
      all: report.txt

      report.txt: plot1.pdf plot2.pdf
        touch report.txt

      plot1.pdf plot2.pdf: plot.py data.txt
        python plot.py  # plot.py produziert sowohl plot1.pdf als auch plot2.pdf
    \end{minted}
  \end{center}

  Lösung: Ein \texttt{make}-Feature (\texttt{v4.3}): \emph{grouped targets}
  \begin{center}
    \small
    \begin{minted}{make}
      all: report.txt

      report.txt: plot1.pdf plot2.pdf
        touch report.txt

      plot2.pdf plot1.pdf &: plot.py data.txt  # das &: definiert die targets als group 
          python plot.py

    \end{minted}
  \end{center}
   → Alle \texttt{targets} werden durch eine einzigen Durchlauf der \texttt{recipe} (gebündelt) erzeugt.
\end{frame}

\begin{frame}[fragile]{Expert}
  Wenn ein Skript sehr viele Dateien erzeugt kann die Liste der \texttt{targets} unübersichtlich lang werden.\\
  Diese \texttt{targets} werden außerdem in der Regel als \texttt{prerequisites} verwendet,
  z.~B. für die \texttt{recipe} der Berichtdatei  → Die Liste befindet sich sogar zweimal im \texttt{Makefile}

  \begin{center}
    \small
    \begin{minted}{make}
      all: report.txt

      report.txt: plot1.pdf plot2.pdf plot3.pdf plot4.pdf plot5.pdf
        touch report.txt

      plot1.pdf plot2.pdf plot3.pdf plot4.pdf plot5.pdf & : plot.py data.txt
        python plot.py  # plot.py produziert alle plot{i}.pdf 
    \end{minted}
  \end{center}
  Lösung: Ein weiteres \texttt{make}-Feature: \emph{variables}
  \begin{center}
    \small
    \begin{minted}{make}
      all: report.txt

      script_targets = plot1.pdf plot2.pdf plot3.pdf plot4.pdf plot5.pdf  # Variablen Definition

      report.txt: $(script_targets)  # Variablen Verwendung
        touch report.txt
      
      $(script_targets) & : plot.py data.txt  # Variablen Verwendung
          python plot.py
    \end{minted}
  \end{center}
\end{frame}

\begin{frame}[fragile]{Expert}
  Die Variablenliste kann auch weiter bearbeitet werden.
  Mit \texttt{addprefix build/} wird in diesem Beispiel der \texttt{build} Ordner
  an den Anfang jedes Dateipfades geschrieben.
  \texttt{addsuffix .pdf} hängt an jeden Dateinamen die Endung \texttt{.pdf} an.
  \begin{center}
    \small
    \begin{minted}{make}
      all: report.txt

      plots = $(addprefix build/, $(addsuffix .pdf, plot1 plot2 plot3 plot4))  # Definition

      report.txt: $(plots)  # Verwendung
        touch report.txt
      
      $(plots) & : plot.py data.txt  # Verwendung
          python plot.py
    \end{minted}
  \end{center}
\end{frame}