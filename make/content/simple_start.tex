\begin{frame}[fragile]{Makefile}
  \begin{itemize}
    \item Von \texttt{make} benutzte Datei heißt \texttt{Makefile} (keine Endung)
      \begin{itemize}
        \item bei Windows Dateiendungen einschalten, siehe
          \url{http://techmixx.de/windows-10-dateiendungen-anzeigen-oder-ausblenden/}
      \end{itemize}
    \item \texttt{Makefile} besteht aus Regeln (Rules):
  \end{itemize}
  \begin{block}{Rule}
    \inputminted{make}{example-files/Basic-block}
  \end{block}
  \begin{description}
    \item[\texttt{\hphantom{prerequisites}\llap{target}}] Datei(en), die von dieser Rule erzeugt werden
    \item[\texttt{prerequisites}]                         Dateien, von denen diese Rule abhängt
    \item[\texttt{\hphantom{prerequisites}\llap{recipe}}] Befehle, um vom \texttt{prerequisites} zu \texttt{target} zu kommen
    \begin{itemize}
      \item wird mit Tab unter \texttt{target: prerequisites} eingerückt
    \end{itemize}
  \end{description}
\end{frame}

\begin{frame}[fragile]{Einfachstes Beispiel}
  \begin{center}
    \inputminted{make}{example-files/Basic-block-example}
  \end{center}
  \begin{itemize}
    \item Wir wollen \texttt{plot.pdf} erzeugen (\texttt{target})
    \begin{itemize}
      \item \texttt{plot.pdf} hängt von \texttt{plot.py} und \texttt{data.txt} ab (\texttt{prerequisites})
      \item Der Befehl, um \texttt{plot.pdf} aus den \texttt{prerequisites} zu erhalten ist \texttt{python plot.py}
    \end{itemize}
  \end{itemize}
\end{frame}

\begin{frame}[fragile]{Beispiel}
  \begin{center}
    \inputminted{make}{example-files/First-report}
  \end{center}
  \vspace{1em}

  \texttt{make} eingeben:
  \begin{itemize}
    \item \texttt{all} braucht \texttt{report.pdf}
      \begin{itemize}
        \item \texttt{report.pdf} braucht \texttt{plot.pdf}
          \begin{itemize}
            \item \texttt{python plot.py}
          \end{itemize}
        \item \texttt{lualatex report.tex}
    \end{itemize}
  \end{itemize}
\end{frame}
