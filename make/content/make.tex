\begin{frame}{Automatisierte, reproduzierbare Prozesse}

  {\huge Problem:}\\
  \vspace{1em}
  Kurz vor Abgabe noch neue Korrekturen einpflegen
  \begin{enumerate}
    \item Tippfehler korrigieren, Plots bearbeiten
    \item \TeX{} ausführen, ausdrucken
  \end{enumerate}
  \begin{itemize}
    \item vergessen, Plots neu zu erstellen
    \item zurück zu Schritt 1 ...  
  \end{itemize}
\end{frame}

\begin{frame}{Automatisierte, reproduzierbare Prozesse}

  {\huge Lösung: Make}
  \vspace{1em}
  \begin{itemize}
    \item prüft, welche Dateien geändert wurden
    \item berechnet nötige Operationen um Abhängigkeiten zu erfüllen
    \item führt Befehle aus
    \begin{itemize}
      \item Python-Skripe
      \item \TeX{}
      \item etc ...
    \end{itemize}
  \end{itemize}
\end{frame}

\begin{frame}{Motivation}
  {\huge Warum?}
  \vspace{1em}
  \begin{itemize}
    \item \textbf{Automatisierung} verhindert Fehler
    \item Dient als \textbf{Dokumentation}
    \item \textbf{Reproduzierbarkeit}: unverzichtbar in der Wissenschaft
    \item \textbf{Spart Zeit}: nur notwendige Operationen werden ausgeführt
  \end{itemize}
  \vspace{1em}
  Idealfall: Eingabe von \texttt{make} erstellt komplettes Protokoll/Paper aus Daten
\end{frame}

\begin{frame}[fragile]{Makefile}
  \begin{itemize}
    \item Von \texttt{make} benutzte Datei heißt \texttt{Makefile} (keine Endung)
      \begin{itemize}
        \item bei Windows Dateiendungen einschalten, siehe \url{http://support.microsoft.com/kb/865219/de}
      \end{itemize}
    \item \texttt{Makefile} besteht aus Regeln (Rules):
  \end{itemize}
  \begin{block}{Rule}
    \centering
    \begin{lstmake}
      target: prerequisites
          recipe
    \end{lstmake}
  \end{block}
  \begin{description}
    \item[\texttt{\hphantom{prerequisites}\llap{target}}] Datei(en), die von dieser Rule erzeugt werden
    \item[\texttt{prerequisites}]                         Dateien, von denen diese Rule abhängt
    \item[\texttt{\hphantom{prerequisites}\llap{recipe}}] Befehle, um vom \texttt{prerequisites} zu \texttt{target} zu kommen
    \begin{itemize}
      \item wird mit Tab unter \texttt{target: prerequisites} eingerückt
    \end{itemize}
  \end{description}
\end{frame}

\begin{frame}[fragile]{Einfachstes Beispiel}
  \begin{center}
    \begin{lstmake}
      plot.pdf: plot.py data.txt
          python plot.py
    \end{lstmake}
  \end{center}
  \begin{itemize}
    \item Wir wollen \texttt{plot.pdf} erzeugen (\texttt{target})
    \begin{itemize}
      \item \texttt{plot.pdf} hängt von \texttt{plot.py} und \texttt{data.txt} ab (\texttt{prerequisites})
      \item Der Befehl, um \texttt{plot.pdf} aus den \texttt{prerequisites} zu erhalten ist \texttt{python plot.pdf}
    \end{itemize}
  \end{itemize}
\end{frame}

\begin{frame}[fragile]{Beispiel}
  \begin{center}
    \begin{lstmake}
      all: report.pdf  # convention

      plot.pdf: plot.py data.txt
          python plot.py

      report.pdf: report.tex
          lualatex report.tex

      report.pdf: plot.pdf  # add prerequisite
    \end{lstmake}
  \end{center}
  \vspace{1em}

  \texttt{make} eingeben:
  \begin{itemize}
    \item \texttt{all} braucht \texttt{report.pdf}
      \begin{itemize}
        \item \texttt{report.pdf} braucht \texttt{plot.pdf}
          \begin{itemize}
            \item \texttt{python plot.py}
          \end{itemize}
        \item \texttt{lualatex report.tex}
    \end{itemize}
  \end{itemize}
\end{frame}

\begin{frame}{Funktionsweise}
  \begin{center}
    \begin{tikzpicture}
      \graph [
        grow down,
        branch right sep,
        nodes={
          node font=\ttfamily,
        },
      ] {
        "all" [x=3.5] -> "report.pdf" [x=3] -> {
          "report.tex",
          "plot.pdf" -> {
            "plot.py", "data.txt"
          },
          "plot2.pdf" -> {
            "plot2.py", "data.txt", "data2.txt"
          },
        },
      };
    \end{tikzpicture}
  \end{center}

  \begin{itemize}
    \item Abhängigkeiten bilden einen DAG (directed acyclic graph)
    \item Dateien werden neu erstellt, falls sie nicht existieren oder älter als ihre Prerequisites sind
    \item Prerequisites werden zuerst erstellt
    \item top-down Vorgehen
  \end{itemize}
\end{frame}

\begin{frame}[fragile]{Argumente für \texttt{make}}
  \begin{tabu}{>{\ttfamily}l X[,L]}
    make \textit{target} & statt des ersten in der \texttt{Makefile} genannten Targets (meist \texttt{all}) nur \texttt{target} erstellen \\
    make -n              & dry run: Befehle anzeigen aber nicht ausführen \\
    make -d              & debug: anzeigen, warum \texttt{make} sich so entschieden hat \\
    make -p              & Datenbank aller Abhängigkeiten ausgeben
  \end{tabu}
  \begin{itemize}
    \item Nützlich, wenn man einen Plot bearbeitet: \texttt{make plot.pdf}
  \end{itemize}
\end{frame}

\begin{frame}[fragile]{\texttt{make clean}}
  (Nützliche) Konvention: \texttt{make clean} löscht alle vom \texttt{Makefile} erstellten Dateien/Ordner.

  \begin{center}
    \begin{lstmake}
      clean:
          rm plot.pdf report.pdf
    \end{lstmake}
  \end{center}

  Das Projekt sollte dann so aussehen, wie vor dem ersten Ausführen von \texttt{make}.
\end{frame}

\begin{frame}[fragile]{Advanced}
  \texttt{build}-Ordner: Projekt sauber halten

  \begin{center}
    \begin{lstmake}
      all: build/report.tex

      build/plot.pdf: plot.py data.txt | build
          python plot.py  # savefig('build/plot.pdf')

      build/report.pdf: report.tex build/plot.pdf | build
          lualatex --output-directory=build report.tex

      build:
          mkdir -p build

      clean:
          rm -rf build

      .PHONY: all clean
    \end{lstmake}
  \end{center}

  \begin{itemize}
    \item \texttt{| build} ist ein order-only Prerequisite: Alter wird ignoriert
    \item Targets, die bei \texttt{.PHONY} genannt werden, entsprechen nicht Dateien (guter Stil)
  \end{itemize}
\end{frame}

\begin{frame}[fragile]{Expert}
  Können mehrere unabhängige Auswertungen parallel ausgeführt werden? \\
  Ja: \texttt{make -j4} (nutzt 4 Prozesse gleichzeitig)

  \begin{center}
    \begin{lstmake}
      plot1.pdf plot2.pdf: plot.py data.txt
          python plot.py
    \end{lstmake}
  \end{center}

  Problem: \texttt{make} führt \texttt{plot.py} gleichzeitig zweimal aus

  Lösung: manuell synchronisieren

  \begin{center}
    \begin{lstmake}
      plot1.pdf: plot.py data.txt
          python plot.py

      plot2.pdf: plot1.pdf
    \end{lstmake}
  \end{center}

  Wenn man \texttt{plot2.pdf} aber nicht \texttt{plot1.pdf} löscht, kann \texttt{make} nicht mehr \texttt{plot2.pdf} erstellen.
\end{frame}
