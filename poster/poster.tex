\RequirePackage{luatex85}
\documentclass{scrartcl}

\usepackage[margin=1cm]{geometry}
\usepackage{fontspec}
\setsansfont{Fira Sans}
\renewcommand{\familydefault}{\sfdefault}

\usepackage{tikz}

\usepackage{xcolor}
\usepackage[colorlinks=true, urlcolor=blue!60!black]{hyperref}

\begin{document}

  \begin{center}
    \fontsize{40}{50}\bfseries\selectfont Toolbox-Workshop 2016
  \end{center}

  \vspace{0.5cm}

  \begin{center}
    \huge Ihr müsst im nächsten Semester Praktikums-Versuche machen? \\[0.5\baselineskip]
    Wir machen es euch einfach(er)!
  \end{center}
  
  \vspace{0.5cm}

  \begin{center}
    \begin{tikzpicture}
      \node[anchor=center] at (0, 3) {%
        \includegraphics[height=2.2cm]{logos/python.pdf}%
        \hspace{1cm}%
        \includegraphics[height=2.2cm]{logos/git.png}%
      };
      \node[anchor=center] at (0,-3) {%
        \includegraphics[height=1.8cm]{logos/matplotlib.png}%
        \hspace{1cm}%
        \includegraphics[height=1.8cm]{logos/numpy.png}%
      };
      \node[anchor=center] at (0, -5.5) {%
        \includegraphics[height=1.2cm]{logos/uncertainties.png}%
        \hspace{1cm}%
        \fontsize{42}{45}\selectfont\texttt{make}%
      };
      \node[anchor=center, align=center] at (0, 0) {%
        \huge Praktikums-Versuche richtig auswerten! \\[0.5\baselineskip]
        \huge 4. – 7.~Oktober 2016, 13–17~Uhr, Chemie HS3
      };
    \end{tikzpicture}
  \end{center}

  \begin{center}
    \huge Wissenschaftliche Arbeiten verfassen mit \\[1\baselineskip]
    \textrm{\fontsize{80}{120}\selectfont\LaTeX{}}\\[1\baselineskip]
    10. – 14.~Oktober 2015, 13–17~Uhr, Chemie HS3
  \end{center}
  \vspace{0.5cm}
  \begin{center}
    \large Weitere Informationen und Umfrage: \\
    \Huge \href{https://toolbox.pep-dortmund.org}{toolbox.pep-dortmund.org}
  \end{center}

\end{document}
