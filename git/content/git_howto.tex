\begin{frame}{Zentrales Konzept: Das Repository}
  \begin{itemize}
    \item Erzeugen mit \texttt{git init}
    \item Damit wird der aktuelle Ordner zu einem Repository
  \end{itemize}
  \vspace{3em}
  \centering
  \begin{tikzpicture}[
      line width=1.5,
      gitstep/.style={
        draw,
        rounded corners,
        thick,
        minimum width=4cm,
        minimum height=1.2cm,
      },
    ]
    \node (wd) at (0, 0) [gitstep, fill=red!20, visible on=<1->, align=center] {Working directory \\{\includegraphics[width=.03\textwidth]{figures/paper.png}}};
    \node (idx) [gitstep, fill=yellow!20, below=0.4cm of wd, visible on=<2->, align=center] {Staging \\{\includegraphics[width=.03\textwidth]{figures/baggage.png}}};
    \node (hist) [gitstep, fill=green!20, below=0.4cm of idx, visible on=<3->, align=center] {History \\{\includegraphics[width=.03\textwidth]{figures/baggage.png}\hspace{0.1cm}\includegraphics[width=.03\textwidth]{figures/baggage.png}\hspace{0.1cm}\includegraphics[width=.03\textwidth]{figures/baggage.png}}};
    \node [right=1cm of wd, text width=0.5\textwidth, align=flush left, visible on=<1->] {Aktuelles Arbeitsverzeichnis, Inhalt des Ordners im Dateisystem.};
    \node [right=1cm of idx, text width=0.5\textwidth, align=flush left, visible on=<2->] {Änderungen, die für einen \enquote{commit} vorgemerkt sind.};
    \node [right=1cm of hist, text width=0.5\textwidth, align=flush left, visible on=<3->] {Gespeicherte \emph{Historie} des Projekts. Alle jemals gemachten Änderungen. Ein Baum von Commits.};
    \draw[thick,->, visible on=<4->] (wd.west) to[out=180,in=180] node[left] {\texttt{git add}} (idx.west);
    \draw[thick,->, visible on=<5->] (idx.west) to[out=180,in=180] node[left] {\texttt{git commit}} (hist.west);
  \end{tikzpicture}
\end{frame}

\begin{frame}{Remotes}
  Remotes sind zentrale Stellen, z.\,B. Server auf denen die History gespeichert wird.
  \begin{center}
  \begin{tikzpicture}[line width=1.5]
    \node (hist) [draw,rounded corners,thick,minimum width=4cm,minimum height=1.2cm,fill=green!20, align=center] {Eure History \\{\includegraphics[width=.03\textwidth]{figures/baggage.png}\hspace{0.1cm}\includegraphics[width=.03\textwidth]{figures/baggage.png}\hspace{0.1cm}\includegraphics[width=.03\textwidth]{figures/baggage.png}}};
    \node (remote) [below=0.4cm of hist,draw,rounded corners,thick,minimum width=4cm,minimum height=1.2cm,fill=red!20, align=center] {Remote \\{\includegraphics[width=.03\textwidth]{figures/baggage.png}\hspace{0.1cm}\includegraphics[width=.03\textwidth]{figures/baggage.png}\hspace{0.1cm}\includegraphics[width=.03\textwidth]{figures/baggage.png}\hspace{0.03\textwidth}\includegraphics[width=.03\textwidth]{figures/baggage.png}}};
    \node (hist2) [visible on=<2->,below=0.4cm of remote,draw,rounded corners,thick,minimum width=4cm,minimum height=1.2cm,fill=blue!20, align=center] {History eures Partners \\{\includegraphics[width=.03\textwidth]{figures/baggage.png}\hspace{0.1cm}\includegraphics[width=.03\textwidth]{figures/baggage.png}\hspace{0.1cm}\includegraphics[width=.03\textwidth]{figures/baggage.png}}};
    \draw[thick,->] (hist.east) to[out=0,in=0] node[right] {\texttt{git push}} (remote.east);
    \draw[thick,->] (remote.west) to[out=180,in=180] node[left] {\texttt{git pull}} (hist.west);
  \end{tikzpicture}
  \end{center}
\end{frame}

\begin{frame}{Konzept: Commits}
  \begin{itemize}
    \item Stand des Repositories zu einem Zeitpunkt
    \item Erlaubt das Hinzufügen von Kommentaren: Was wurde getan seit dem letzten Commit?
    \item Sind die \textit{Versionen} des Repositories
  \end{itemize}
  \begin{tikzpicture}
    \node (commit1) at (0, 0) {{\includegraphics[width=0.2\textwidth]{figures/baggage.png}}};
    \node (commit2) at (8, 0) {{\includegraphics[width=0.2\textwidth]{figures/baggage.png}}};
    \node (commit1_text) [below=0.5cm of commit1] {\texttt{Commit 1}};
    \node (commit2_text) [below=0.5cm of commit2] {\texttt{Commit 2}};
    \draw[arr] (commit1) -- (commit2);
  \end{tikzpicture}
\end{frame}

\begin{frame}{History}
  \only<1>{
    \begin{tikzpicture}
      \graph [
        grow right=1.5cm,
        node distance=0.5cm,
        nodes={
          blue!70!black,
          node font=\ttfamily,
        },
      ]{
        "Erstabgabe" [visible on=<4->, green!60!black, x=4];
		a <- b <- c <- d <- main [vertexDarkRed];
        f[visible on=<2>] ->[visble on=<2>] -> a;
        "Erstabgabe"[visible on=<2>] ->[visible on=<3>] c;
      };
    \end{tikzpicture}
  }
  \only<2>{
    \begin{tikzpicture}
      \graph [
        grow right=1.5cm,
        node distance=0.5cm,
        nodes={
          blue!70!black,
          node font=\ttfamily,
        },
      ]{
        "Erstabgabe" [visible on=<4->, green!60!black, x=4];
        a <- b <- c <- {
          d <- main [vertexDarkRed],
          e <- "fix\_plot" [vertexDarkRed]
        };
        "Erstabgabe"[visible on=<3>] ->[visible on=<3>] c;
      };
    \end{tikzpicture}
  }
  \only<3->{
    \begin{tikzpicture}
      \graph [
        grow right=1.5cm,
        nodes={
          blue!70!black,
          node font=\ttfamily,
        },
      ] {
        "Erstabgabe" [visible on=<4->, green!60!black, x=6],
        a <- b <- c <- {
          d <- f,
          e <- g [x=0.5],
        } <- h <- {i <-[visible on=<2->] main [vertexDarkRed, visible on=<3->], j <-[visible on=<2->] "fix\_formula" [vertexDarkRed, visible on=<3->]};
        "Erstabgabe" ->[visible on=<4->] f;
      };
    \end{tikzpicture}
  }

  \vspace{1em}
  \begin{itemize}
    \item<1-> \textcolor{blue}{Commit}: Zustand/Inhalt des Arbeitsverzeichnisses zu einem Zeitpunkt
      \begin{itemize}
        \item Enthält Commit-Message (Beschreibung der Änderungen)
        \item Wird über einen Hash-Code identifiziert
        \item Zeigt immer auf seine(n) Vorgänger
      \end{itemize}
    \item<2-> \textcolor{vertexDarkRed}{Branch}: benannter Zeiger auf einen Commit
      \begin{itemize}
        \item Entwicklungszweig
		\item Im Praktikum reicht meist der Standard-Branch: \texttt{main}
		    \item Wandert weiter mit dem aktuellsten Commit
      \end{itemize}
    \item<4-> \textcolor{green!60!black}{Tag}: unveränderbarer Zeiger auf einen Commit
      \begin{itemize}
        \item Wichtiges Ereignis, z.B. veröffentlichte Version
      \end{itemize}
  \end{itemize}
\end{frame}

\begin{frame}{Typischer Arbeitsablauf}
  \begin{enumerate}
    \item Neues Repo? Repository erzeugen oder klonen: \hfill\texttt{git init}, \texttt{git clone} \\
      Repo schon da? Änderungen herunterladen: \hfill\texttt{git pull}
    \item Arbeiten
      \begin{enumerate}
        \item Dateien bearbeiten und testen
        \item Änderungen vorbereiten: \hfill\texttt{git add}
        \item Änderungen als \emph{commit} speichern: \hfill\texttt{git commit}
      \end{enumerate}
    \item Commits anderer herunterladen und integrieren: \hfill\texttt{git pull}
    \item Eigene Commits hochladen: \hfill\texttt{git push}
  \end{enumerate}
\end{frame}

\begin{frame}{Zum selber ausprobieren und Lernen:}
  \begin{figure}
    \centering
    \includegraphics[width=.6\textwidth]{figures/learngitbranching.png}
  \end{figure}
  \texttt{https://learngitbranching.js.org/}
\end{frame}
