\begin{frame}[fragile]{Achtung: Merge conflicts}
  \begin{center}
    \huge Don't Panic
  \end{center}

  Entstehen, wenn \texttt{git} nicht automatisch mergen kann (selbe Zeile geändert, etc.)

  \begin{enumerate}
    \item Die betroffenen Dateien öffnen
    \item Markierungen finden und die Stelle selbst mergen (meist wenige Zeilen)
      \begin{verbatim}
        <<<<<<< HEAD
        foo
        ||||||| merged common ancestors
        bar
        =======
        baz
        >>>>>>> Commit-Message
\end{verbatim}
    \item Merge abschließen:
      \begin{enumerate}
          \item \texttt{git add …} \quad (Files mit behobenen Konflikten)
        \item \texttt{git commit} \quad\rightarrow\; Editor wird geöffnet
        \item Vorgeschlagene Nachricht kann angenommen werden (In vim \texttt{":wq"} eintippen)
      \end{enumerate}
  \end{enumerate}
  Nützlich: \texttt{git config --global merge.conflictstyle zdiff3}
\end{frame}

\begin{frame}{Zu früheren Versionen zurückkehren}
  \begin{tblr}{
      colspec = {l X[l]},
      column{1} = {font=\ttfamily},
    }
    git checkout \textit{commit} & Commit ins Arbeitsverzeichnis laden \\
    git restore \textit{filename} & Änderungen an Dateien verwerfen (zum letzten Commit zurückkehren)
  \end{tblr}
\end{frame}

\begin{frame}{Kurz an was anderem Arbeiten}
  \begin{tblr}{
      colspec = {l X[l]},
      column{1} = {font=\ttfamily},
    }
    git stash     & Änderungen kurz zur Seite schieben \\
    git stash pop & Änderungen zurückholen aus Stash
  \end{tblr}
\end{frame}

\begin{frame}[fragile]{\texttt{.gitignore}}
    \begin{itemize}
    \item Man möchte nicht alle Dateien von \texttt{git} beobachten lassen
    \item z.B. \texttt{build}-Ordner
    \end{itemize}
    \begin{center}
        \Large Lösung: \texttt{.gitignore}-Datei
    \end{center}

    \begin{itemize}
    \item einfache Textdatei
    \item enthält Regeln für Dateien, die nicht beobachtet werden sollen
    \end{itemize}
    Beispiel:
    \vspace{1em}
    \begin{minted}{text}
      build/
      *.pdf
      **/__pycache__/  # recursion-trick
    \end{minted}
\end{frame}
