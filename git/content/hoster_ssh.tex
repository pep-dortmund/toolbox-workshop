\begin{frame}{Hoster}
  \begin{tblr}{
      colspec = {X[c] X[c] X[c]},
      measure=vbox,
    }
    \href{https://github.com}{\includegraphics[width=0.75\linewidth]{figures/github.png}} &
    \href{https://bitbucket.org}{\includegraphics[width=0.75\linewidth]{figures/bitbucket.png}} &
    \href{https://gitlab.com}{\includegraphics[width=0.75\linewidth]{figures/gitlab.png}} \\
    \begin{itemize}
      \item größter Hoster
      \item viele open-source Projekte
      \item Unbegrenzt private Repositories für Studenten und Forscher:  \newline
        \href{http://education.github.com}{education.github.com}
    \end{itemize}
    &
    \begin{itemize}
      \item kostenlose private Repos mit höchstens fünf Leuten
      \item keine Speicherbegrenzungen
      \item Hängt was Oberfläche und Funktionen angeht, den beiden anderen weit hinterher
    \end{itemize}
    &
    \begin{itemize}
      \item open-source
      \item keine Begrenzungen an privaten Repos
      \item kann man selbst auf einem eigenen Server betreiben
    \end{itemize}\\
  \end{tblr}

  Weitere Open Source Optionen, auch zum selbst hosten: Gitea, Forgejo

  \begin{center}
    \onslide<2->{%
      \Large%
      \enquote{Now, everybody sort of gets born with a GitHub account}
      – Guido van Rossum
    }
  \end{center}
\end{frame}

\begin{frame}[fragile]{SSH-Keys}
  Git kann auf mehrere Arten mit einem Server kommunizieren:
  \begin{description}
    \item[HTTPS]
      \begin{itemize}
        \item Mit Nutzername / Passwort:
          War lange die einfachste Möglichkeit. Wird aber von GitHub aus Sicherheitsgründen nicht mehr einfach unterstützt.
        \item Mit \enquote{Personal Access Token}. Neues Verfahren für GitHub über HTTPS.
      \end{itemize}
    \item[SSH]: Keys müssen erzeugt und eingestellt werden, Passwort für den Key muss, wenn ein \enquote{SSH-Agent} verwendet wird, nur einmal pro Session eingegeben werden.
  \end{description}

  SSH-Keys:
  \begin{enumerate}
    \item \mintinline{text}+ssh-keygen -t ed25519 -C "your_email@example.com"+
    \item Namen wählen, z.\,B.\ \mintinline{text}+~/.ssh/github_key_ed25519+;
      Passwort wählen
    \item \mintinline{text}+cat ~/.ssh/github_key_ed25519.pub+
      \qquad (Public-Key ausgeben)
    \item Ausgabe ist Public-Key, beim Server eintragen (im Browser)
    \item \mintinline{text}+ssh-add ~/.ssh/github_key_ed25519+
      \qquad (Schlüssel zum Schlüsselring hinzufügen)
  \end{enumerate}

  Für den Agent, falls noch nicht vom Betriebsystem eingerichtet (z.\,B.\ Windows mit WSL):
  \begin{enumerate}
    \setcounter{enumi}{4}
    \item \mintinline{text}+echo 'eval $(ssh-agent -s)' >> ~/.bashrc+
    \item \mintinline{text}+echo 'AddKeysToAgent yes' >> ~/.ssh/config+
  \end{enumerate}

  Doku: \url{https://docs.github.com/en/authentication/connecting-to-github-with-ssh}
\end{frame}
