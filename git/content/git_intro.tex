\headlineframe{Lösung: Änderungen verwalten mit \texttt{git}}

\begin{frame}
    \centering
    \includegraphics[width=0.7\textwidth]{logos/git_logo.pdf}

    \vspace{1em}

    \begin{itemize}
      \item Ein Versionskontrollsystem
      \item Ursprünglich entwickelt, um den Programmcode des Linux-Kernels zu verwalten (Linus Torvalds)
      \item Hat sich gegenüber ähnlichen Programmen (SVN, mercurial) durchgesetzt
      \item Wird in der Regel über die Kommandozeile benutzt
      \item Es gibt auch Plugins für Editoren, z.B. VS Code
    \end{itemize}
\end{frame}

\begin{frame}{Was bringt \texttt{git} für Vorteile?}
  \begin{itemize}
    \item Arbeit wird für andere sichtbar protokolliert
    \item Erlaubt Zurückspringen an einen früheren Zeitpunkt
    \item Kann die meisten Änderungen automatisch zusammenfügen
    \item Wirkt nebenbei auch als Backup
  \end{itemize}
  Einzige Herausforderung: Man muss lernen, damit umzugehen.
\end{frame}
