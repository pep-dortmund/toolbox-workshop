\newcommand{\mail}{\includegraphics[width=0.2\textwidth]{figures/mail.pdf}}
\tikzstyle{arr}=[line width=1mm,draw=black,-triangle 45]

\begin{frame}[c]{Was ist Versionskontrolle?}
  \begin{itemize}
    \item Versionskontrollsoftware speichert Änderungen an Dokumenten / Dateien
    \item Das kann fast alles sein:
      \begin{itemize}
        \item Software
        \item Rechtliche Dokumente
        \item Dokumentation
        \item Wissenschaftliche Veröffentlichungen
        \item Bilder
        \item Baupläne, CAD-Zeichnungen
        \item ...
      \end{itemize}
    \item Ein \emph{Schnappschuss} eines Projektes nennt man \emph{Revision}.
    \item Alle Revisionen zusammen bilden die \emph{Geschichte} des Projekts.
  \end{itemize}
\end{frame}

\begin{frame}[c]{Warum also Versionskontrolle nutzen?}
  \begin{itemize}
    \item Erlaubt, an eine beliebige Revision zurückzukehren
    \item Kann die Unterschiede zwischen Revisionen anzeigen
    \item Erleichtert die Zusammenarbeit an Projekten
    \item Dient auch als Backup
  \end{itemize}
\end{frame}

\begin{frame}[c]{Warum also Versionskontrolle nutzen?}
  Versionskontrollsoftware macht die Beantwortung der folgenden Fragen einfach:
  \begin{description}[Warum?]
    \item[Was?] Was wurde von Revision \emph{A} auf Revision \emph{B} geändert
    \item[Wer?] Wer hat eine Änderung gemacht? Wer hat alles zum Projekt beigetragen?
    \item[Warum?] Warum wurde diese Änderung gemacht?
    \item[Wann?] Wann wurde ein bestimmter Bug eingeführt bzw. behoben?
  \end{description}

  \onslide<2>{%
    \begin{center}%
      \Large\color{vertexDarkRed}%
      Versionskontrolle ist eine fundamentale Bedingung\\
      für nachvollziehbare, reproduzierbare Wissenschaft.%
    \end{center}
  }
\end{frame}
