\newcommand{\mail}{\includegraphics[width=0.2\textwidth]{figures/mail.pdf}}
\tikzstyle{arr}=[line width=1mm,draw=black,-triangle 45]

\headlineframe{Wie arbeitet man am besten an einem Protokoll zusammen?}

\headlineframe{Idee: Austausch über Mails}

\begin{frame}{Mails: Probleme}
  \centering
  \begin{tikzpicture}
    \node (alice) at (0, 0) {\includegraphics[width=0.2\textwidth]{figures/text.pdf}};
    \node (bob) at (8, 0) {\includegraphics[width=0.2\textwidth]{figures/text.pdf}};
    \draw[arr] (alice) -- node {\mail} (bob);
  \end{tikzpicture}
  \begin{itemize}
    \item Risiko, dass Änderungen vergessen werden, ist groß
    \item Bei jedem Abgleich muss jemand anders aktiv werden
      \begin{itemize}
        \item Stört
        \item Es kommt zu Verzögerungen
      \end{itemize}
  \end{itemize}
  \textbf{\Large Fazit: Eine sehr unbequeme / riskante Lösung}
\end{frame}

\headlineframe{Idee: Austausch über Dropbox}

\begin{frame}{Dropbox: Probleme}
  \centering
  \begin{tikzpicture}
    \node (alice) at (0, 0) {\includegraphics[width=0.2\textwidth]{figures/text.pdf}};
    \node[align=center,fill=blue!20,rounded corners=1em] (dropbox) at (4, 0.5) {
      \includegraphics[width=0.1\textwidth]{logos/dropbox.pdf}\\
      \includegraphics[width=0.2\textwidth]{figures/text.pdf}
    };
    \node (bob) at (8, 0) {\includegraphics[width=0.2\textwidth]{figures/text.pdf}};
    \draw[arr,triangle 45-triangle 45] (0.8, 0) -- (dropbox);
    \draw[arr,triangle 45-triangle 45] (dropbox) -- (7.2, 0);
  \end{tikzpicture}

  \begin{itemize}
    \item Man merkt nichts von Änderungen der Anderen
    \item Gleichzeitige Änderungen führen zu \enquote{In Konflikt sehende Kopie}-Dateien.
    \item Änderungen werden nicht zusammengeführt.
  \end{itemize}
  \textbf{\Large Fazit: Besser, aber hat deutliche Probleme}
\end{frame}

\headlineframe{Lösung: Änderungen verwalten mit \texttt{git}}

\begin{frame}
    \centering
    \includegraphics[width=0.7\textwidth]{logos/git.pdf}

    \vspace{1em}

    \begin{itemize}
      \item Ein Versionskontrollsystem
      \item Ursprünglich entwickelt, um den Programmcode des Linux-Kernels zu verwalten (Linus Torvalds)
      \item Hat sich gegenüber ähnlichen Programmen (SVN, mercurial) durchgesetzt
    \end{itemize}
\end{frame}

\begin{frame}{Was bringt \texttt{git} für Vorteile?}
  \begin{itemize}
    \item Arbeit wird für andere sichtbar protokolliert
    \item Erlaubt Zurückspringen an einen früheren Zeitpunkt
    \item Kann die meisten Änderungen automatisch zusammenfügen
    \item Wirkt nebenbei auch als Backup
  \end{itemize}
  Einziges Problem: Man muss lernen, damit umzugehen
\end{frame}

\begin{frame}{Zentrales Konzept: Das Repository}
  \begin{itemize}
    \item Erzeugen mit \texttt{git init}
  \end{itemize}
  \vspace{3em}
  \centering
  \begin{tikzpicture}[line width=1.5]
    \node (wd) at (0, 0) [draw,rounded corners,thick,minimum width=4cm,minimum height=1.2cm,fill=red!20] {Working directory};
    \node (idx) [below=0.4cm of wd,draw,thick,rounded corners,minimum width=4cm,minimum height=1.2cm,fill=yellow!20] {Index};
    \node (hist) [below=0.4cm of idx,draw,rounded corners,thick,minimum width=4cm,minimum height=1.2cm,fill=green!20] {History};
    \draw[thick,->] (wd.east) to[out=0,in=0] node[right] {\texttt{git add}} (idx.east);
    \draw[thick,->] (idx.west) to[out=180,in=180] node[left] {\texttt{git commit}} (hist.west);
  \end{tikzpicture}
\end{frame}

\begin{frame}{Mit anderen Repositories kommunizieren}
  \begin{itemize}
    \item Repository kopieren: \texttt{git clone}
    \item Neue Änderungen holen: \texttt{git pull}
    \item Eigene Änderungen hochladen: \texttt{git push}
  \end{itemize}
\end{frame}

\begin{frame}{Achtung: Merge conflicts}
  \begin{center}
    \huge Don't Panic
  \end{center}
  \begin{itemize}
    \item Entstehen, wenn \texttt{git} nicht automatisch mergen kann (selbe Zeile geändert, etc.)
  \end{itemize}
  \begin{enumerate}
    \item Die betroffenen Dateien öffnen
    \item Markierungen finden und die Stelle selbst mergen (meist 2,3 Zeilen)
    \item \texttt{git commit} ausführen um zu bestätigen
  \end{enumerate}
\end{frame}

\begin{frame}{Weitere nützliche Befehle}
  \begin{itemize}
    \item Änderungen ansehen: \texttt{git diff}
    \item Vergangenheit betrachten: \texttt{git log}
    \item Änderungen kurz zur Seite schieben: \texttt{git stash}
  \end{itemize}
\end{frame}

