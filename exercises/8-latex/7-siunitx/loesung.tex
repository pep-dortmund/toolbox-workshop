\documentclass[titlepage=firstiscover]{scrartcl}

\usepackage{fixltx2e}
\usepackage[aux]{rerunfilecheck}

\usepackage{polyglossia}
\setmainlanguage{german}

\usepackage{fontspec}

\usepackage[
  locale=DE,
  separate-uncertainty=true,
  per-mode=symbol-or-fraction,
]{siunitx}

\usepackage[unicode]{hyperref}
\usepackage{bookmark}

\begin{document}

\section{Aufgabe 2}
Das Newton ist definiert als die Kraft, die eine Masse von \SI{1}{\kilo\gram} in einer Sekunde auf die Geschwindigkeit \SI{1}{\meter\per\second} beschleunigt:
$\si{\newton} = \si{\kilo\gram\meter\per\second\squared}$.

Als Gleichung:
\begin{equation}
  \si{\newton} = \si{\kilo\gram\meter\per\second\squared} .
\end{equation}

\section{Aufgabe 3}

\begin{equation}
  \num{5.2(1)} = \num{5.2 +- 0.1}
\end{equation}

\end{document}
