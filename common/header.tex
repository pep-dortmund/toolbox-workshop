\PassOptionsToPackage{unicode, pdfusetitle, colorlinks, linkcolor=TUgreen, urlcolor=TUorange}{hyperref}
\documentclass[9pt, mathserif, professionalfonts]{beamer}

\usetheme{TUDo}

\usepackage[ngerman]{babel}
\usepackage[backend=biber,sorting=none]{biblatex}

\usepackage{fontspec}

\setsansfont{Tex Gyre Heros}

\setlength{\parindent}{0pt}
\usepackage[german=quotes]{csquotes}
\usepackage{eurosym}

\usepackage{amsmath}
\usepackage{mathtools}
\usepackage{amssymb}
\usepackage{xfrac}
\usepackage{unicode-math}
\setmathfont{Tex Gyre Pagella Math}

\usepackage[locale=DE, separate-uncertainty=true, per-mode=fraction]{siunitx}
\AtBeginDocument{\sisetup{math-rm=\mathrm, text-rm=\rmfamily}}

\usepackage{booktabs}
\usepackage{tabulary}
\usepackage{threeparttable}

\usepackage[]{graphicx}
%\usepackage[labelfont=bf]{caption}
\setbeamertemplate{caption}[numbered]

\usepackage{xcolor}
\usepackage{enumerate}

\usepackage{listings}
\usepackage{lstautogobble}

\lstnewenvironment{lstverbatim}[1][]{
  \lstset
  {
    autogobble,
    language={[LaTeX]TeX},
    basicstyle=\ttfamily,
    texcsstyle=*\color{TUdarkgreen},
    keywordstyle=\color{TUorange},
    commentstyle=\color{blue},
    morekeywords = {hallo},
    otherkeywords={$,$, &, \{, \}, \[, \],  ^, _},
    moretexcs={setmainlanguage, text, num, SI, si, kilo, gram, befehl, subject, titlehead, publishers,
               metre, per, second, squared, micro, subtitle, maketitle, tableofcontents,
               subsection, subsubsection, part, chapter, paragraph, subparagraph,
               iint, iiint},
    columns=fullflexible,
    breaklines=true,
    keepspaces=true,
    aboveskip=-0.2em,
    belowskip=-0.2em
  }
}{}

\usepackage{tikz}
\usetikzlibrary{shadows,arrows,shapes,positioning}

\tikzstyle{buttonstyle} = [align=center, rectangle, fill = black!10, draw = black!80, drop shadow, font={\bfseries}, text=TUgreen]

\newcommand{\button}[1]{\tikz[baseline={([yshift=-.8ex]current bounding box.center)}]{\node[buttonstyle] {#1};}}

\newcommand{\doc}[2]{\button{\href{#1}{Doku: #2}}}

\newcommand{\headlineframe}[1]{  % used to show headline at beginning of each logical section
    \begin{frame}[plain]
        \begin{center}
            \Huge\color{TUgreen}#1
        \end{center}
    \end{frame}
}

\author{PeP et al. Toolbox}
\date{2014}
\institute[Pep et al. e.V.]{\includegraphics[width=0.2\textwidth]{logos/pep.pdf}}
