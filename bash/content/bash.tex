\headlineframe{Unix Shell}

\begin{frame}{Dateisystem}
  \begin{itemize}
    \item bildet \emph{einen} Baum
      \begin{itemize}
        \item beginnt bei \texttt{/} (root)
        \item \texttt{/} trennt Teile eines Pfads
        \item auf Groß-/Kleinschreibung achten!
      \end{itemize}
    \item es gibt ein aktuelles Verzeichnis
    \item relative vs. absolute Pfade
    \item spezielle Verzeichnisse:
      \begin{itemize}
        \item \texttt{.} das aktuelle Verzeichnis
        \item \texttt{..} das Oberverzeichnis
        \item \texttt{\textasciitilde} das Homeverzeichnis
      \end{itemize}
  \end{itemize}
\end{frame}

\begin{frame}{\texttt{man}, \texttt{pwd}, \texttt{cd}}
  \begin{tabular}{lp{20em}}
    \texttt{man \textit{topic}} & \enquote{manual}: zeigt die Hilfe für ein Programm \\
    \texttt{pwd} & \enquote{print working directory}: zeigt das aktuelle Verzeichnis \\
    \texttt{cd \textit{directory}} & \enquote{change directory}: wechselt in das angegebene Verzeichnis
  \end{tabular}
\end{frame}

\begin{frame}{\texttt{ls}}
  \begin{tabular}{lp{20em}}
    \texttt{ls [\textit{directory}]} & \enquote{list}: zeigt den Inhalt eines Verzeichnisses an \\
    \texttt{ls -l} & \enquote{long}: zeigt mehr Informationen über Dateien und Verzeichnisse \\
    \texttt{ls -a} & \enquote{all}: zeigt auch versteckte Dateien (fangen mit \texttt{.} an)
  \end{tabular}
\end{frame}

\begin{frame}{\texttt{mkdir}, \texttt{touch}}
  \begin{tabular}{lp{17em}}
    \texttt{mkdir \textit{directory}} & \enquote{make directory}: erstellt ein neues Verzeichnis \\
    \texttt{mkdir -p \textit{directory}} & \enquote{parent}: erstellt auch alle notwendigen Oberverzeichnisse \\
    \texttt{touch \textit{file}} & erstellt eine leere Datei
  \end{tabular}
\end{frame}

\begin{frame}{\texttt{cp}, \texttt{mv}, \texttt{rm}, \texttt{rmdir}}
  \begin{tabular}{lp{15em}}
    \texttt{cp \textit{source} \textit{destination}} & \enquote{copy}: kopiert eine Datei \\
    \texttt{cp -r \textit{source} \textit{destination}} & \enquote{recursive}: kopiert ein Verzeichnis rekursiv \\
    \texttt{mv \textit{source} \textit{desination}} & \enquote{move}: verschiebt eine Datei (Umbenennung) \\
    \texttt{rm \textit{file}} & \enquote{remove}: löscht eine Datei (Es gibt keinen Papierkorb!) \\
    \texttt{rm -r \textit{directory}} & \enquote{recursive}: löscht ein Verzeichnis rekursiv \\
    \texttt{rmdir \textit{directory}} & \enquote{remove directory}: löscht ein \emph{leeres} Verzeichnis
  \end{tabular}
\end{frame}

\begin{frame}{\texttt{cat}, \texttt{less}, \texttt{grep}, \texttt{echo}}
  \begin{tabular}{lp{15em}}
    \texttt{cat \textit{file}} & \enquote{concatenate}: gibt Inhalt einer (oder mehr) Datei(en) aus \\
    \texttt{less \texttt{file}} & (besser als \texttt{more}): wie \texttt{cat}, aber navigabel \\
    \texttt{grep \textit{pattern} \textit{file}} & \texttt{g/re/p}: sucht in einer Datei nach einem Muster \\
    \texttt{grep -i \textit{pattern} \textit{file}} & \enquote{case insensitive} \\
    \texttt{grep -r \textit{pattern} \textit{directory}} & \enquote{recursive}: suche rekursiv in allen Dateien \\
    \texttt{echo \textit{message}} & gibt einen Text aus
  \end{tabular}
\end{frame}

\begin{frame}{Ein- und Ausgabe}
  \begin{tabular}{lp{20em}}
    \texttt{\textit{command} > \textit{file}} & überschreibt Datei mit Ausgabe \\
    \texttt{\textit{command} >> \textit{file}} & fügt Ausgabe einer Datei hinzu \\
    \texttt{\textit{command} < \textit{file}} & Datei als Eingabe \\
    \texttt{\textit{command1} | \textit{command2}} & Ausgabe als Eingabe (Pipe)
  \end{tabular}
\end{frame}

\begin{frame}{Tastaturkürzel}
  \begin{tabular}{lp{20em}}
    \texttt{Ctrl-C} & beendet das laufende Programm \\
    \texttt{Ctrl-D} & EOF (end of file) eingeben, kann Programme beenden \\
    \texttt{Ctrl-L} & leert den Bildschirm
  \end{tabular}
\end{frame}

\begin{frame}{Globbing}
  \begin{tabular}{lp{20em}}
    \texttt{*} & wird ersetzt durch alle passenden Dateien \\
    \texttt{\{\textit{a},\textit{b}\}} & bildet alle Kombinationen
  \end{tabular}

  \vspace{2cm}
  Beispiele:\\
  \begin{tabular}{lcl}
    \texttt{*.log} & → & \texttt{foo.log bar.log} \\
    \texttt{foo.\{tex,pdf\}} & → & \texttt{foo.tex foo.pdf}
  \end{tabular}
\end{frame}
