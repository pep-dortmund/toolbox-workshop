\section{Durchführung}
\label{sec:Durchführung}

Wie in der Anleitung~\cite{V16516} beschrieben, wird der Versuch wie folgt durchgeführt.

Wie einst von Galileo Galilei \cite[232]{galilei1623} persönlich, wurden im hier beschriebenen Versuch
Objekte (eine Kugel und  Holzylinder in Form eines Glases) eine schiefe Ebene hinab gerollt.

Notiert wird dabei die Starthöhen $h$. Der Prozess des Herabrollens wird mit einer Kamera 
aufgezeichnet, um aus der Zeit $t$ bis zum erreichen des Endes der schiefen Ebene möglichst genau
aus den Einzelbildern (frames) des Films ablesen zu können.

Die aufgenommenen Messdaten werden für zwei unabhängige Zwecke verwendet:

\begin{enumerate}
  \item {Bestimmung der Fallbeschleunigung $g$, dafür werden 
        die theoretischen Trägheitsmomente angenommen.}
      \item {Bestimmung der Trägheitsmomente $I$ der beiden Objekte unter Annahme der theoretischen 
        Fallbeschleunigung.}
\end{enumerate}


